\documentclass[]{book}
\usepackage{lmodern}
\usepackage{amssymb,amsmath}
\usepackage{ifxetex,ifluatex}
\usepackage{fixltx2e} % provides \textsubscript
\ifnum 0\ifxetex 1\fi\ifluatex 1\fi=0 % if pdftex
  \usepackage[T1]{fontenc}
  \usepackage[utf8]{inputenc}
\else % if luatex or xelatex
  \ifxetex
    \usepackage{mathspec}
  \else
    \usepackage{fontspec}
  \fi
  \defaultfontfeatures{Ligatures=TeX,Scale=MatchLowercase}
\fi
% use upquote if available, for straight quotes in verbatim environments
\IfFileExists{upquote.sty}{\usepackage{upquote}}{}
% use microtype if available
\IfFileExists{microtype.sty}{%
\usepackage{microtype}
\UseMicrotypeSet[protrusion]{basicmath} % disable protrusion for tt fonts
}{}
\usepackage{hyperref}
\hypersetup{unicode=true,
            pdftitle={Data Wrangling with R},
            pdfauthor={Claudia A Engel},
            pdfborder={0 0 0},
            breaklinks=true}
\urlstyle{same}  % don't use monospace font for urls
\usepackage{natbib}
\bibliographystyle{apalike}
\usepackage{color}
\usepackage{fancyvrb}
\newcommand{\VerbBar}{|}
\newcommand{\VERB}{\Verb[commandchars=\\\{\}]}
\DefineVerbatimEnvironment{Highlighting}{Verbatim}{commandchars=\\\{\}}
% Add ',fontsize=\small' for more characters per line
\usepackage{framed}
\definecolor{shadecolor}{RGB}{248,248,248}
\newenvironment{Shaded}{\begin{snugshade}}{\end{snugshade}}
\newcommand{\AlertTok}[1]{\textcolor[rgb]{0.94,0.16,0.16}{#1}}
\newcommand{\AnnotationTok}[1]{\textcolor[rgb]{0.56,0.35,0.01}{\textbf{\textit{#1}}}}
\newcommand{\AttributeTok}[1]{\textcolor[rgb]{0.77,0.63,0.00}{#1}}
\newcommand{\BaseNTok}[1]{\textcolor[rgb]{0.00,0.00,0.81}{#1}}
\newcommand{\BuiltInTok}[1]{#1}
\newcommand{\CharTok}[1]{\textcolor[rgb]{0.31,0.60,0.02}{#1}}
\newcommand{\CommentTok}[1]{\textcolor[rgb]{0.56,0.35,0.01}{\textit{#1}}}
\newcommand{\CommentVarTok}[1]{\textcolor[rgb]{0.56,0.35,0.01}{\textbf{\textit{#1}}}}
\newcommand{\ConstantTok}[1]{\textcolor[rgb]{0.00,0.00,0.00}{#1}}
\newcommand{\ControlFlowTok}[1]{\textcolor[rgb]{0.13,0.29,0.53}{\textbf{#1}}}
\newcommand{\DataTypeTok}[1]{\textcolor[rgb]{0.13,0.29,0.53}{#1}}
\newcommand{\DecValTok}[1]{\textcolor[rgb]{0.00,0.00,0.81}{#1}}
\newcommand{\DocumentationTok}[1]{\textcolor[rgb]{0.56,0.35,0.01}{\textbf{\textit{#1}}}}
\newcommand{\ErrorTok}[1]{\textcolor[rgb]{0.64,0.00,0.00}{\textbf{#1}}}
\newcommand{\ExtensionTok}[1]{#1}
\newcommand{\FloatTok}[1]{\textcolor[rgb]{0.00,0.00,0.81}{#1}}
\newcommand{\FunctionTok}[1]{\textcolor[rgb]{0.00,0.00,0.00}{#1}}
\newcommand{\ImportTok}[1]{#1}
\newcommand{\InformationTok}[1]{\textcolor[rgb]{0.56,0.35,0.01}{\textbf{\textit{#1}}}}
\newcommand{\KeywordTok}[1]{\textcolor[rgb]{0.13,0.29,0.53}{\textbf{#1}}}
\newcommand{\NormalTok}[1]{#1}
\newcommand{\OperatorTok}[1]{\textcolor[rgb]{0.81,0.36,0.00}{\textbf{#1}}}
\newcommand{\OtherTok}[1]{\textcolor[rgb]{0.56,0.35,0.01}{#1}}
\newcommand{\PreprocessorTok}[1]{\textcolor[rgb]{0.56,0.35,0.01}{\textit{#1}}}
\newcommand{\RegionMarkerTok}[1]{#1}
\newcommand{\SpecialCharTok}[1]{\textcolor[rgb]{0.00,0.00,0.00}{#1}}
\newcommand{\SpecialStringTok}[1]{\textcolor[rgb]{0.31,0.60,0.02}{#1}}
\newcommand{\StringTok}[1]{\textcolor[rgb]{0.31,0.60,0.02}{#1}}
\newcommand{\VariableTok}[1]{\textcolor[rgb]{0.00,0.00,0.00}{#1}}
\newcommand{\VerbatimStringTok}[1]{\textcolor[rgb]{0.31,0.60,0.02}{#1}}
\newcommand{\WarningTok}[1]{\textcolor[rgb]{0.56,0.35,0.01}{\textbf{\textit{#1}}}}
\usepackage{longtable,booktabs}
\usepackage{graphicx,grffile}
\makeatletter
\def\maxwidth{\ifdim\Gin@nat@width>\linewidth\linewidth\else\Gin@nat@width\fi}
\def\maxheight{\ifdim\Gin@nat@height>\textheight\textheight\else\Gin@nat@height\fi}
\makeatother
% Scale images if necessary, so that they will not overflow the page
% margins by default, and it is still possible to overwrite the defaults
% using explicit options in \includegraphics[width, height, ...]{}
\setkeys{Gin}{width=\maxwidth,height=\maxheight,keepaspectratio}
\IfFileExists{parskip.sty}{%
\usepackage{parskip}
}{% else
\setlength{\parindent}{0pt}
\setlength{\parskip}{6pt plus 2pt minus 1pt}
}
\setlength{\emergencystretch}{3em}  % prevent overfull lines
\providecommand{\tightlist}{%
  \setlength{\itemsep}{0pt}\setlength{\parskip}{0pt}}
\setcounter{secnumdepth}{5}
% Redefines (sub)paragraphs to behave more like sections
\ifx\paragraph\undefined\else
\let\oldparagraph\paragraph
\renewcommand{\paragraph}[1]{\oldparagraph{#1}\mbox{}}
\fi
\ifx\subparagraph\undefined\else
\let\oldsubparagraph\subparagraph
\renewcommand{\subparagraph}[1]{\oldsubparagraph{#1}\mbox{}}
\fi

%%% Use protect on footnotes to avoid problems with footnotes in titles
\let\rmarkdownfootnote\footnote%
\def\footnote{\protect\rmarkdownfootnote}

%%% Change title format to be more compact
\usepackage{titling}

% Create subtitle command for use in maketitle
\providecommand{\subtitle}[1]{
  \posttitle{
    \begin{center}\large#1\end{center}
    }
}

\setlength{\droptitle}{-2em}

  \title{Data Wrangling with R}
    \pretitle{\vspace{\droptitle}\centering\huge}
  \posttitle{\par}
    \author{Claudia A Engel}
    \preauthor{\centering\large\emph}
  \postauthor{\par}
      \predate{\centering\large\emph}
  \postdate{\par}
    \date{Last updated: October 08, 2019}

\usepackage{booktabs}
\usepackage{amsthm}
\makeatletter
\def\thm@space@setup{%
  \thm@preskip=8pt plus 2pt minus 4pt
  \thm@postskip=\thm@preskip
}
\makeatother

\begin{document}
\maketitle

{
\setcounter{tocdepth}{1}
\tableofcontents
}
\hypertarget{prerequisites-and-preparations}{%
\chapter*{Prerequisites and Preparations}\label{prerequisites-and-preparations}}
\addcontentsline{toc}{chapter}{Prerequisites and Preparations}

\begin{itemize}
\tightlist
\item
  You should have some \textbf{basic knowledge} of R, and be familiar with the topics covered in the \href{https://cengel.github.io/R-intro/}{Introduction to R}.
\item
  Have a recent version of \href{https://cran.r-project.org/}{R} and \href{https://www.rstudio.com/}{RStudio} installed.
\item
  Install and load the \texttt{tidyverse} package.
\end{itemize}

\begin{Shaded}
\begin{Highlighting}[]
\KeywordTok{install.packages}\NormalTok{(}\StringTok{"tidyverse"}\NormalTok{)  }
\KeywordTok{library}\NormalTok{(tidyverse)}
\end{Highlighting}
\end{Shaded}

\begin{itemize}
\tightlist
\item
  Create a new RStudio project \texttt{R-data-ws} in a new folder \texttt{R-data-ws}. Download both CSV files into a subdirectory called \texttt{data} like this:
\item
  Download \texttt{MS\_trafficstops\_bw\_age.csv}:
\end{itemize}

\begin{Shaded}
\begin{Highlighting}[]
\KeywordTok{download.file}\NormalTok{(}\StringTok{"http://bit.ly/MS_trafficstops_bw_age"}\NormalTok{,}
              \StringTok{"data/MS_trafficstops_bw_age.csv"}\NormalTok{)}
\end{Highlighting}
\end{Shaded}

\begin{itemize}
\tightlist
\item
  Download \texttt{MS\_acs2015\_bw.csv}:
\end{itemize}

\begin{Shaded}
\begin{Highlighting}[]
\KeywordTok{download.file}\NormalTok{(}\StringTok{"http://bit.ly/MS_acs_2015_bw"}\NormalTok{,}
              \StringTok{"data/MS_acs2015_bw.csv"}\NormalTok{)}
\end{Highlighting}
\end{Shaded}

\hypertarget{references}{%
\section*{References}\label{references}}
\addcontentsline{toc}{section}{References}

Boehmke, Bradley C. (2016) Data Wrangling with R
\url{http://link.springer.com/book/10.1007\%2F978-3-319-45599-0}

Grolemund, G \& Wickham, H (2017): R for Data Science \url{http://r4ds.had.co.nz}

Wickham, H. (2014): Tidy Data \url{https://www.jstatsoft.org/article/view/v059i10}

\hypertarget{acknowledgements}{%
\section*{Acknowledgements}\label{acknowledgements}}
\addcontentsline{toc}{section}{Acknowledgements}

Part of the materials for this tutorial are adapted from \url{http://datacarpentry.org} and \url{http://softwarecarpentry.org}.

\hypertarget{dplyr}{%
\chapter{\texorpdfstring{Data Manipulation using \textbf{\texttt{dplyr}}}{Data Manipulation using dplyr}}\label{dplyr}}

\begin{quote}
Learning Objectives

\begin{itemize}
\tightlist
\item
  Select columns in a data frame with the \textbf{\texttt{dplyr}} function \texttt{select}.
\item
  Select rows in a data frame according to filtering conditions with the \textbf{\texttt{dplyr}} function \texttt{filter}.
\item
  Direct the output of one \textbf{\texttt{dplyr}} function to the input of another function with the `pipe' operator \texttt{\%\textgreater{}\%}.
\item
  Add new columns to a data frame that are functions of existing columns with \texttt{mutate}.
\item
  Understand the split-apply-combine concept for data analysis.
\item
  Use \texttt{summarize}, \texttt{group\_by}, and \texttt{tally} to split a data frame into groups of observations, apply a summary statistics for each group, and then combine the results.
\end{itemize}
\end{quote}

\begin{center}\rule{0.5\linewidth}{\linethickness}\end{center}

We will be working a small subset of the data from the \href{https://openpolicing.stanford.edu}{Stanford Open Policing Project}. It contains information about traffic stops for blacks and whites in the state of Mississippi during January 2013 to mid-July of 2016.

Let's begin with loading our sample data into a data frame.

\begin{Shaded}
\begin{Highlighting}[]
\NormalTok{trafficstops <-}\StringTok{ }\KeywordTok{read.csv}\NormalTok{(}\StringTok{"data/MS_trafficstops_bw_age.csv"}\NormalTok{)}
\end{Highlighting}
\end{Shaded}

Manipulation of dataframes is a common task when you start exploring your data. We might select certain observations (rows) or variables (columns), group the data by a certain variable(s), or calculate summary statistics.

If we were interested in the mean age of the driver in different counties we can do this using the normal base R operations:

\begin{Shaded}
\begin{Highlighting}[]
\KeywordTok{mean}\NormalTok{(trafficstops[trafficstops}\OperatorTok{$}\NormalTok{county_name }\OperatorTok{==}\StringTok{ "Clay County"}\NormalTok{, }\StringTok{"driver_age"}\NormalTok{], }
     \DataTypeTok{na.rm =} \OtherTok{TRUE}\NormalTok{)}
\end{Highlighting}
\end{Shaded}

\begin{verbatim}
#> [1] 31.8002
\end{verbatim}

\begin{Shaded}
\begin{Highlighting}[]
\KeywordTok{mean}\NormalTok{(trafficstops[trafficstops}\OperatorTok{$}\NormalTok{county_name }\OperatorTok{==}\StringTok{ "Lee County"}\NormalTok{, }\StringTok{"driver_age"}\NormalTok{], }
     \DataTypeTok{na.rm =} \OtherTok{TRUE}\NormalTok{)}
\end{Highlighting}
\end{Shaded}

\begin{verbatim}
#> [1] 34.66915
\end{verbatim}

\begin{Shaded}
\begin{Highlighting}[]
\KeywordTok{mean}\NormalTok{(trafficstops[trafficstops}\OperatorTok{$}\NormalTok{county_name }\OperatorTok{==}\StringTok{ "Yazoo County"}\NormalTok{, }\StringTok{"driver_age"}\NormalTok{], }
     \DataTypeTok{na.rm =} \OtherTok{TRUE}\NormalTok{)}
\end{Highlighting}
\end{Shaded}

\begin{verbatim}
#> [1] 37.05759
\end{verbatim}

Bracket subsetting is handy, but it can be cumbersome and difficult to read, especially for complicated operations. Furthermore, there is a fair amount of repetition. Repeating yourself will cost you time, both now and later, and potentially introduce some nasty bugs.

\textbf{\texttt{dplyr}} is a package for making tabular data manipulation easier.

\begin{quote}
Brief recap:
Packages in R are sets of additional functions that let you do more stuff. Functions like \texttt{str()} or \texttt{data.frame()}, come built into R; packages give you access to more of them. Before you use a package for the first time you need to install it on your machine, and then you should import it in every subsequent R session when you need it.
\end{quote}

If you haven't, please installe the \textbf{\texttt{tidyverse}} package.

\begin{Shaded}
\begin{Highlighting}[]
\KeywordTok{install.packages}\NormalTok{(}\StringTok{"tidyverse"}\NormalTok{)    }
\end{Highlighting}
\end{Shaded}

\textbf{\texttt{tidyverse}} is an ``umbrella-package'' that installs a series of packages useful for data analysis which work together well. Some of them are considered \textbf{core} packages (among them \textbf{\texttt{tidyr}}, \textbf{\texttt{dplyr}}, \textbf{\texttt{ggplot2}}), because you are likely to use them in almost every analysis. Other packages, like \texttt{lubridate} (to work wiht dates) or \texttt{haven} (for SPSS, Stata, and SAS data) that you are likely to use not for every analysis are also installed.

If you type the following command, it will load the \textbf{core} \texttt{tidyverse} packages.

\begin{Shaded}
\begin{Highlighting}[]
\KeywordTok{library}\NormalTok{(}\StringTok{"tidyverse"}\NormalTok{)    }\CommentTok{## load the core tidyverse packages, incl. dplyr}
\end{Highlighting}
\end{Shaded}

If you need to use functions from \texttt{tidyverse} packages other than the core packages, you will need to load them separately.

\hypertarget{what-is-dplyr}{%
\section{\texorpdfstring{What is \textbf{\texttt{dplyr}}?}{What is dplyr?}}\label{what-is-dplyr}}

\textbf{\texttt{dplyr}} is one part of a larger \textbf{\texttt{tidyverse}} that enables you to work
with data in tidy data formats. ``Tidy datasets are easy to manipulate, model and visualise, and have a specific structure: each variable is a column, each observation is a row, and each type of observational unit is a table.'' (From Wickham, H. (2014): Tidy Data \url{https://www.jstatsoft.org/article/view/v059i10})

The package \textbf{\texttt{dplyr}} provides convenient tools for the most common data manipulation
tasks. It is built to work directly with data frames, with many common tasks
optimized by being written in a compiled language (C++). An additional feature is the
ability to work directly with data stored in an external database. The benefits of
doing this are that the data can be managed natively in a relational database,
queries can be conducted on that database, and only the results of the query are
returned.

This addresses a common problem with R in that all operations are conducted
in-memory and thus the amount of data you can work with is limited by available
memory. The database connections essentially remove that limitation in that you
can have a database of many 100s GB, conduct queries on it directly, and pull
back into R only what you need for analysis.

To learn more about \textbf{\texttt{dplyr}} after the workshop, you may want to check out the \href{https://github.com/rstudio/cheatsheets/raw/master/data-transformation.pdf}{handy data transformation with \textbf{\texttt{dplyr}} cheatsheet}.

\hypertarget{subsetting-columns-and-rows}{%
\section{Subsetting columns and rows}\label{subsetting-columns-and-rows}}

To select columns of a
data frame with \texttt{dplyr}, use \texttt{select()}. The first argument to this function is the data
frame (\texttt{trafficstops}), and the subsequent arguments are the columns to keep.

\begin{Shaded}
\begin{Highlighting}[]
\KeywordTok{select}\NormalTok{(trafficstops, police_department, officer_id, driver_race)}
\end{Highlighting}
\end{Shaded}

\begin{verbatim}
#>            police_department officer_id driver_race
#> 1 Mississippi Highway Patrol       J042       Black
#> 2 Mississippi Highway Patrol       B026       Black
#> 3 Mississippi Highway Patrol       M009       Black
#> 4 Mississippi Highway Patrol       K035       White
#> 5 Mississippi Highway Patrol       D028       White
#> 6 Mississippi Highway Patrol       K023       White
\end{verbatim}

It is worth knowing that \texttt{dplyr} comes with a number of \href{https://www.rdocumentation.org/packages/dplyr/versions/0.7.2/topics/select_helpers}{``select helpers''}, which are functions that allow you to select columns based on their names. For example:

\begin{Shaded}
\begin{Highlighting}[]
\KeywordTok{select}\NormalTok{(trafficstops, }\KeywordTok{starts_with}\NormalTok{(}\StringTok{"driver"}\NormalTok{))}
\end{Highlighting}
\end{Shaded}

\begin{verbatim}
#>   driver_gender driver_birthdate driver_race driver_age
#> 1          male       1950-06-14       Black         63
#> 2          male       1967-04-06       Black         46
#> 3          male       1974-04-15       Black         39
#> 4          male       1981-03-23       White         32
#> 5          male       1992-08-03       White         20
#> 6        female       1960-05-02       White         53
\end{verbatim}

To choose rows based on specific criteria, use \texttt{filter()}:

\begin{Shaded}
\begin{Highlighting}[]
\KeywordTok{filter}\NormalTok{(trafficstops, county_name }\OperatorTok{==}\StringTok{ "Yazoo County"}\NormalTok{)}
\end{Highlighting}
\end{Shaded}

\begin{verbatim}
#>              id state  stop_date  county_name county_fips
#> 1 MS-2013-00252    MS 2013-01-02 Yazoo County       28163
#> 2 MS-2013-00253    MS 2013-01-02 Yazoo County       28163
#> 3 MS-2013-00254    MS 2013-01-02 Yazoo County       28163
#> 4 MS-2013-00331    MS 2013-01-02 Yazoo County       28163
#> 5 MS-2013-00350    MS 2013-01-02 Yazoo County       28163
#> 6 MS-2013-00426    MS 2013-01-03 Yazoo County       28163
#>            police_department driver_gender driver_birthdate driver_race
#> 1 Mississippi Highway Patrol          male       1950-05-04       Black
#> 2 Mississippi Highway Patrol        female       1967-05-29       Black
#> 3 Mississippi Highway Patrol          male       1986-12-21       Black
#> 4 Mississippi Highway Patrol        female       1986-02-01       Black
#> 5 Mississippi Highway Patrol          male       1994-11-21       White
#> 6 Mississippi Highway Patrol          male       1994-02-24       White
#>                                                 violation_raw officer_id
#> 1 Speeding - Regulated or posted speed limit and actual speed       C037
#> 2 Speeding - Regulated or posted speed limit and actual speed       C011
#> 3 Speeding - Regulated or posted speed limit and actual speed       C011
#> 4 Speeding - Regulated or posted speed limit and actual speed       C037
#> 5 Speeding - Regulated or posted speed limit and actual speed       C037
#> 6 Speeding - Regulated or posted speed limit and actual speed       C014
#>   driver_age violation
#> 1         63  Speeding
#> 2         46  Speeding
#> 3         26  Speeding
#> 4         27  Speeding
#> 5         18  Speeding
#> 6         19  Speeding
\end{verbatim}

Here are some other ways to select rows:

\begin{itemize}
\tightlist
\item
  select certain rows by row number: \texttt{slice(trafficstops,\ 1:3)\ \#\ rows\ 1-3}
\item
  select random rows:

  \begin{itemize}
  \tightlist
  \item
    \texttt{sample\_n(trafficstops,\ 5)\ \#\ number\ of\ rows\ to\ select}
  \item
    \texttt{sample\_frac(trafficstops,\ .01)\ \#\ fraction\ of\ rows\ to\ select}
  \end{itemize}
\end{itemize}

To sort rows by variables use the \texttt{arrange} function: \texttt{arrange(trafficstops,\ county\_name,\ stop\_date)}

\begin{verbatim}
#>              id state  stop_date  county_name county_fips
#> 1 MS-2013-07659    MS 2013-02-09 Adams County       28001
#> 2 MS-2013-11819    MS 2013-03-02 Adams County       28001
#> 3 MS-2013-14647    MS 2013-03-16 Adams County       28001
#> 4 MS-2013-15430    MS 2013-03-20 Adams County       28001
#> 5 MS-2013-18581    MS 2013-04-06 Adams County       28001
#> 6 MS-2013-20016    MS 2013-04-13 Adams County       28001
#>            police_department driver_gender driver_birthdate driver_race
#> 1 Mississippi Highway Patrol          male       1989-06-12       Black
#> 2 Mississippi Highway Patrol        female       1974-10-16       Black
#> 3 Mississippi Highway Patrol        female       1977-07-15       Black
#> 4 Mississippi Highway Patrol        female       1991-06-15       Black
#> 5 Mississippi Highway Patrol        female       1980-04-18       White
#> 6 Mississippi Highway Patrol        female       1996-01-14       Black
#>                                                 violation_raw officer_id
#> 1 Speeding - Regulated or posted speed limit and actual speed       M004
#> 2                             Driving while license suspended       M042
#> 3 Speeding - Regulated or posted speed limit and actual speed       M049
#> 4            Failure to maintain required liability insurance       M049
#> 5            Failure to maintain required liability insurance       M010
#> 6 Speeding - Regulated or posted speed limit and actual speed       M024
#>   driver_age                violation
#> 1         24                 Speeding
#> 2         38 License-Permit-Insurance
#> 3         36                 Speeding
#> 4         22 License-Permit-Insurance
#> 5         33 License-Permit-Insurance
#> 6         17                 Speeding
\end{verbatim}

\hypertarget{pipes}{%
\section{Pipes}\label{pipes}}

What if you wanted to filter \textbf{and} select on the same data? For example, lets find drivers over 85 years and only keep the violation and gender columns. There are three ways to do this: use intermediate steps, nested functions, or pipes.

\begin{itemize}
\tightlist
\item
  Intermediate steps:
\end{itemize}

With intermediate steps, you essentially create a temporary data frame and use
that as input to the next function. This can clutter up your workspace with lots
of objects.

\begin{Shaded}
\begin{Highlighting}[]
\NormalTok{tmp_df <-}\StringTok{ }\KeywordTok{filter}\NormalTok{(trafficstops, driver_age }\OperatorTok{>}\StringTok{ }\DecValTok{85}\NormalTok{)}
\KeywordTok{select}\NormalTok{(tmp_df, violation_raw, driver_gender)}
\end{Highlighting}
\end{Shaded}

\begin{itemize}
\tightlist
\item
  Nested functions
\end{itemize}

You can also nest functions (i.e.~one function inside of another).
This is handy, but can be difficult to read if too many functions are nested as things are evaluated from the inside out.

\begin{Shaded}
\begin{Highlighting}[]
\KeywordTok{select}\NormalTok{(}\KeywordTok{filter}\NormalTok{(trafficstops, driver_age }\OperatorTok{>}\StringTok{ }\DecValTok{85}\NormalTok{), violation_raw, driver_gender)}
\end{Highlighting}
\end{Shaded}

\begin{itemize}
\tightlist
\item
  Pipes!
\end{itemize}

The last option, pipes, are a fairly recent addition to R. Pipes let you take
the output of one function and send it directly to the next, which is useful
when you need to do many things to the same dataset. Pipes in R look like
\texttt{\%\textgreater{}\%} and are made available via the \texttt{magrittr} package, installed automatically
with \textbf{\texttt{dplyr}}. If you use RStudio, you can type the pipe with Ctrl
+ Shift + M if you have a PC or Cmd +
Shift + M if you have a Mac.

\begin{Shaded}
\begin{Highlighting}[]
\NormalTok{trafficstops }\OperatorTok
\StringTok{  }\KeywordTok{filter}\NormalTok{(driver_age }\OperatorTok{>}\StringTok{ }\DecValTok{85}\NormalTok{) }\OperatorTok
\StringTok{  }\KeywordTok{select}\NormalTok{(violation_raw, driver_gender)}
\end{Highlighting}
\end{Shaded}

In the above, we use the pipe to send the \texttt{trafficstops} dataset first through
\texttt{filter()} to keep rows where \texttt{driver\_race} is Black, then through \texttt{select()}
to keep only the \texttt{officer\_id} and \texttt{stop\_date} columns. Since \texttt{\%\textgreater{}\%} takes
the object on its left and passes it as the first argument to the function on
its right, we don't need to explicitly include it as an argument to the
\texttt{filter()} and \texttt{select()} functions anymore.

If we wanted to create a new object with this smaller version of the data, we
could do so by assigning it a new name:

\begin{Shaded}
\begin{Highlighting}[]
\NormalTok{senior_drivers <-}\StringTok{ }\NormalTok{trafficstops }\OperatorTok
\StringTok{  }\KeywordTok{filter}\NormalTok{(driver_age }\OperatorTok{>}\StringTok{ }\DecValTok{85}\NormalTok{) }\OperatorTok
\StringTok{  }\KeywordTok{select}\NormalTok{(violation_raw, driver_gender, driver_race)}

\NormalTok{senior_drivers}
\end{Highlighting}
\end{Shaded}

\begin{verbatim}
#>                                                 violation_raw
#> 1                     Seat belt not used properly as required
#> 2 Speeding - Regulated or posted speed limit and actual speed
#> 3                     Seat belt not used properly as required
#>   driver_gender driver_race
#> 1          male       White
#> 2          male       White
#> 3          male       Black
\end{verbatim}

Note that the final data frame is the leftmost part of this expression.

\begin{quote}
Challenge

Using pipes, subset the \texttt{trafficstops} data to include stops in Tunica County only and retain the columns \texttt{stop\_date}, \texttt{driver\_age}, and \texttt{violation\_raw}. Bonus: sort the table by driver age.
\end{quote}

\hypertarget{add-new-columns}{%
\section{Add new columns}\label{add-new-columns}}

Frequently you'll want to create new columns based on the values in existing columns. For this we'll use \texttt{mutate()}.

To create a new column with the year the driver was born we can extract the first 4 elements of the string that represents the \texttt{driver\_birthdate} and add it to the data frame like this:

\begin{Shaded}
\begin{Highlighting}[]
\NormalTok{trafficstops }\OperatorTok\StringTok{ }
\StringTok{  }\KeywordTok{mutate}\NormalTok{(}\DataTypeTok{birth_year =} \KeywordTok{substring}\NormalTok{(driver_birthdate, }\DecValTok{1}\NormalTok{, }\DecValTok{4}\NormalTok{))}
\end{Highlighting}
\end{Shaded}

The new and edited columns will not permanently be added to the existing data frame -- unless we explicitly save the output.

We can als use \texttt{mutate} to reassign values to an existing column. So here is an alternative to do the same as above, but here we first will use the \texttt{lubridate} library, which is installed with \texttt{tidyverse} to convert our string to an actual date format. We will then use the \texttt{year()} function to extract the year. All of this can happen in the same call to \texttt{mutate()}.

\begin{Shaded}
\begin{Highlighting}[]
\KeywordTok{library}\NormalTok{(lubridate)}

\NormalTok{trafficstops }\OperatorTok\StringTok{ }
\StringTok{  }\KeywordTok{mutate}\NormalTok{(}\DataTypeTok{birth_date =} \KeywordTok{ymd}\NormalTok{(driver_birthdate),}
         \DataTypeTok{birth_year =} \KeywordTok{year}\NormalTok{(driver_birthdate))}
\end{Highlighting}
\end{Shaded}

If this runs off your screen and you just want to see the first few rows, you
can use a pipe to view the \texttt{head()} of the data. (Pipes work with non-\textbf{\texttt{dplyr}}
functions, too, as long as the \textbf{\texttt{dplyr}} or \texttt{magrittr} package is loaded). When piping into a function with no additional arguments, you can call the
function with or without parentheses (e.g.~\texttt{head} or \texttt{head()}). (I like to add the parentheses to remind myself that it is a function and not a variable.)

\begin{Shaded}
\begin{Highlighting}[]
\NormalTok{trafficstops }\OperatorTok\StringTok{ }
\StringTok{  }\KeywordTok{mutate}\NormalTok{(}\DataTypeTok{birth_date =} \KeywordTok{ymd}\NormalTok{(driver_birthdate),}
         \DataTypeTok{birth_year =} \KeywordTok{year}\NormalTok{(driver_birthdate)) }\OperatorTok\StringTok{ }
\StringTok{  }\KeywordTok{head}\NormalTok{()}
\end{Highlighting}
\end{Shaded}

We can keep adding columns like this:

\begin{Shaded}
\begin{Highlighting}[]
\NormalTok{trafficstops }\OperatorTok\StringTok{ }
\StringTok{  }\KeywordTok{mutate}\NormalTok{(}\DataTypeTok{birth_date =} \KeywordTok{ymd}\NormalTok{(driver_birthdate),}
         \DataTypeTok{birth_year =} \KeywordTok{year}\NormalTok{(driver_birthdate),}
         \DataTypeTok{birth_cohort =} \KeywordTok{round}\NormalTok{(birth_year}\OperatorTok{/}\DecValTok{10}\NormalTok{)}\OperatorTok{*}\DecValTok{10}\NormalTok{) }\OperatorTok\StringTok{ }
\StringTok{  }\KeywordTok{head}\NormalTok{()}
\end{Highlighting}
\end{Shaded}

We are beginning to see the power of piping. Here is a slightly expanded example, where we select the column \texttt{birth\_cohort} that we have created and send it to plot:

\begin{Shaded}
\begin{Highlighting}[]
\NormalTok{trafficstops }\OperatorTok\StringTok{ }
\StringTok{  }\KeywordTok{mutate}\NormalTok{(}\DataTypeTok{birth_date =} \KeywordTok{ymd}\NormalTok{(driver_birthdate),}
         \DataTypeTok{birth_year =} \KeywordTok{year}\NormalTok{(driver_birthdate),}
         \DataTypeTok{birth_cohort =} \KeywordTok{round}\NormalTok{(birth_year}\OperatorTok{/}\DecValTok{10}\NormalTok{)}\OperatorTok{*}\DecValTok{10}\NormalTok{,}
         \DataTypeTok{birth_cohort =} \KeywordTok{factor}\NormalTok{(birth_cohort)) }\OperatorTok
\StringTok{    }\KeywordTok{select}\NormalTok{(birth_cohort) }\OperatorTok\StringTok{ }
\StringTok{    }\KeywordTok{plot}\NormalTok{()}
\end{Highlighting}
\end{Shaded}

\begin{figure}
\centering
\includegraphics{R-data-wrangling_files/figure-latex/driver-birth-cohorts-1.pdf}
\caption{\label{fig:driver-birth-cohorts}Driver Birth Cohorts}
\end{figure}

\begin{quote}
Challenge

Create a new data frame from the \texttt{trafficstops} data that meets the following
criteria: contains only the \texttt{violation\_raw} column for female drivers of age 50 that were stopped on a Sunday. For this add a new column to your data frame called
\texttt{weekday\_of\_stop} containing the number of the weekday when the stop occurred. Use the \texttt{wday()} function from \texttt{lubridate} (Sunday = 1).

Think about how the commands should be ordered to produce this data frame!
\end{quote}

\hypertarget{what-is-split-apply-combine}{%
\section{What is split-apply-combine?}\label{what-is-split-apply-combine}}

Many data analysis tasks can be approached using the \emph{split-apply-combine}
paradigm: split the data into groups, apply some analysis to each group, and
then combine the results.

\begin{figure}
\includegraphics[width=\textwidth]{img/split-apply-combine} \caption{Split - Apply - Combine}\label{fig:split-apply-combine}
\end{figure}

\textbf{\texttt{dplyr}} makes this very easy through the use of the
\texttt{group\_by()} function.

\texttt{group\_by()} is often used together with \texttt{summarize()}, which collapses each
group into a single-row summary of that group. \texttt{group\_by()} takes as arguments
the column names that contain the \textbf{categorical} variables for which you want
to calculate the summary statistics. So to view the mean age for black and white drivers:

\begin{Shaded}
\begin{Highlighting}[]
\NormalTok{trafficstops }\OperatorTok
\StringTok{  }\KeywordTok{group_by}\NormalTok{(driver_race) }\OperatorTok
\StringTok{  }\KeywordTok{summarize}\NormalTok{(}\DataTypeTok{mean_age =} \KeywordTok{mean}\NormalTok{(driver_age, }\DataTypeTok{na.rm=}\OtherTok{TRUE}\NormalTok{))}
\end{Highlighting}
\end{Shaded}

\begin{verbatim}
#> Warning: Factor `driver_race` contains implicit NA, consider using
#> `forcats::fct_explicit_na`
\end{verbatim}

\begin{verbatim}
#> # A tibble: 3 x 2
#>   driver_race mean_age
#>   <fct>          <dbl>
#> 1 Black           34.2
#> 2 White           36.2
#> 3 <NA>            34.5
\end{verbatim}

So, back to the beginning of the chapter, where we tried to calculate the mean age of the driver for different counties, how would we do this? Like this:

\begin{Shaded}
\begin{Highlighting}[]
\NormalTok{trafficstops }\OperatorTok
\StringTok{  }\KeywordTok{group_by}\NormalTok{(county_name) }\OperatorTok
\StringTok{  }\KeywordTok{summarize}\NormalTok{(}\DataTypeTok{mean_age =} \KeywordTok{mean}\NormalTok{(driver_age, }\DataTypeTok{na.rm=}\OtherTok{TRUE}\NormalTok{))}
\end{Highlighting}
\end{Shaded}

You can also group by multiple columns:

\begin{Shaded}
\begin{Highlighting}[]
\NormalTok{trafficstops }\OperatorTok\StringTok{ }
\StringTok{  }\KeywordTok{group_by}\NormalTok{(county_name, driver_race) }\OperatorTok
\StringTok{  }\KeywordTok{summarize}\NormalTok{(}\DataTypeTok{mean_age =} \KeywordTok{mean}\NormalTok{(driver_age, }\DataTypeTok{na.rm=}\OtherTok{TRUE}\NormalTok{))}
\end{Highlighting}
\end{Shaded}

\begin{verbatim}
#> Warning: Factor `driver_race` contains implicit NA, consider using
#> `forcats::fct_explicit_na`
\end{verbatim}

\begin{verbatim}
#> # A tibble: 178 x 3
#> # Groups:   county_name [82]
#>    county_name   driver_race mean_age
#>    <fct>         <fct>          <dbl>
#>  1 Adams County  Black           36.2
#>  2 Adams County  White           40.0
#>  3 Alcorn County Black           34.6
#>  4 Alcorn County White           33.6
#>  5 Amite County  Black           37.5
#>  6 Amite County  White           42.1
#>  7 Amite County  <NA>            24  
#>  8 Attala County Black           36.4
#>  9 Attala County White           38.6
#> 10 Benton County Black           34.7
#> # ... with 168 more rows
\end{verbatim}

If we wanted to remove the line with \texttt{NA} we could insert a \texttt{filter()} in the chain:

\begin{Shaded}
\begin{Highlighting}[]
\NormalTok{trafficstops }\OperatorTok
\StringTok{  }\KeywordTok{filter}\NormalTok{(}\OperatorTok{!}\KeywordTok{is.na}\NormalTok{(driver_race)) }\OperatorTok\StringTok{ }
\StringTok{  }\KeywordTok{group_by}\NormalTok{(county_name, driver_race) }\OperatorTok
\StringTok{  }\KeywordTok{summarize}\NormalTok{(}\DataTypeTok{mean_age =} \KeywordTok{mean}\NormalTok{(driver_age, }\DataTypeTok{na.rm=}\OtherTok{TRUE}\NormalTok{))}
\end{Highlighting}
\end{Shaded}

\begin{verbatim}
#> # A tibble: 163 x 3
#> # Groups:   county_name [82]
#>    county_name   driver_race mean_age
#>    <fct>         <fct>          <dbl>
#>  1 Adams County  Black           36.2
#>  2 Adams County  White           40.0
#>  3 Alcorn County Black           34.6
#>  4 Alcorn County White           33.6
#>  5 Amite County  Black           37.5
#>  6 Amite County  White           42.1
#>  7 Attala County Black           36.4
#>  8 Attala County White           38.6
#>  9 Benton County Black           34.7
#> 10 Benton County White           32.0
#> # ... with 153 more rows
\end{verbatim}

Recall that \texttt{is.na()} is a function that determines whether something is an \texttt{NA}. The \texttt{!} symbol negates the result, so we're asking for everything that is \emph{not} an \texttt{NA}.

You may have noticed that the output from these calls looks a little different. That's because \textbf{\texttt{dplyr}} has changed our \texttt{data.frame} object
to an object of class \texttt{tbl\_df}, also known as a ``tibble''. Tibble's data
structure is very similar to a data frame. For our purposes the only differences
are that (1) columns of class \texttt{character} are never converted into
factors, and (2) in addition to displaying the data type of each column under its name, it only prints the first few rows of data and only as many columns as fit on one screen. If we wanted to print all columns we can use the print command, and set the \texttt{width} parameter to \texttt{Inf}. To print the first 6 rows for example we would do this: \texttt{print(my\_tibble,\ n=6,\ width=Inf)}.

Once the data are grouped, you can also summarize multiple variables at the same
time (and not necessarily on the same variable). For instance, we could add a
column indicating the minimum age in each group (i.e.~county):

\begin{Shaded}
\begin{Highlighting}[]
\NormalTok{trafficstops }\OperatorTok
\StringTok{  }\KeywordTok{filter}\NormalTok{(}\OperatorTok{!}\KeywordTok{is.na}\NormalTok{(driver_race)) }\OperatorTok\StringTok{ }
\StringTok{  }\KeywordTok{group_by}\NormalTok{(county_name, driver_race) }\OperatorTok
\StringTok{  }\KeywordTok{summarize}\NormalTok{(}\DataTypeTok{mean_age =} \KeywordTok{mean}\NormalTok{(driver_age, }\DataTypeTok{na.rm=}\OtherTok{TRUE}\NormalTok{),}
            \DataTypeTok{min_age =} \KeywordTok{min}\NormalTok{(driver_age, }\DataTypeTok{na.rm=}\OtherTok{TRUE}\NormalTok{))}
\end{Highlighting}
\end{Shaded}

\begin{verbatim}
#> # A tibble: 163 x 4
#> # Groups:   county_name [82]
#>    county_name   driver_race mean_age min_age
#>    <fct>         <fct>          <dbl>   <int>
#>  1 Adams County  Black           36.2      16
#>  2 Adams County  White           40.0      16
#>  3 Alcorn County Black           34.6      17
#>  4 Alcorn County White           33.6      15
#>  5 Amite County  Black           37.5      17
#>  6 Amite County  White           42.1      15
#>  7 Attala County Black           36.4       8
#>  8 Attala County White           38.6      15
#>  9 Benton County Black           34.7      18
#> 10 Benton County White           32.0      18
#> # ... with 153 more rows
\end{verbatim}

\hypertarget{tallying}{%
\section{Tallying}\label{tallying}}

When working with data, it is also common to want to know the number of
observations found for each factor or combination of factors. For this, \textbf{\texttt{dplyr}}
provides \texttt{tally()}. For example, if we wanted to see how many traffic stops each officer recorded we would do:

\begin{Shaded}
\begin{Highlighting}[]
\NormalTok{trafficstops }\OperatorTok
\StringTok{  }\KeywordTok{group_by}\NormalTok{(officer_id) }\OperatorTok
\StringTok{  }\KeywordTok{tally}\NormalTok{()}
\end{Highlighting}
\end{Shaded}

We can optionally sort the results in descending order by adding \texttt{sort=TRUE}:

\begin{Shaded}
\begin{Highlighting}[]
\NormalTok{trafficstops }\OperatorTok
\StringTok{  }\KeywordTok{group_by}\NormalTok{(officer_id) }\OperatorTok
\StringTok{  }\KeywordTok{tally}\NormalTok{(}\DataTypeTok{sort=}\OtherTok{TRUE}\NormalTok{)}
\end{Highlighting}
\end{Shaded}

Here, \texttt{tally()} is the action applied to the groups created by \texttt{group\_by()} and counts the total number of records for each category.

Alternatives:

\begin{Shaded}
\begin{Highlighting}[]
\NormalTok{trafficstops }\OperatorTok
\StringTok{  }\KeywordTok{group_by}\NormalTok{(officer_id) }\OperatorTok
\StringTok{  }\KeywordTok{summarize}\NormalTok{(}\DataTypeTok{n =} \KeywordTok{n}\NormalTok{()) }\CommentTok{# n() is useful when count is needed for a calculation}
\end{Highlighting}
\end{Shaded}

\begin{verbatim}
#> Warning: Factor `officer_id` contains implicit NA, consider using
#> `forcats::fct_explicit_na`
\end{verbatim}

\begin{Shaded}
\begin{Highlighting}[]
\NormalTok{trafficstops }\OperatorTok
\StringTok{  }\KeywordTok{count}\NormalTok{(officer_id) }\CommentTok{# count() calls group_by automatically, then tallies}
\end{Highlighting}
\end{Shaded}

\begin{verbatim}
#> Warning: Factor `officer_id` contains implicit NA, consider using
#> `forcats::fct_explicit_na`
\end{verbatim}

\begin{quote}
Challenge

Which 5 counties were the ones with the most stops in 2013?
Hint: use the year() function from lubridate.
\end{quote}

\hypertarget{joining-two-tables}{%
\section{Joining two tables}\label{joining-two-tables}}

\textless\textless\textless\textless\textless\textless\textless{} HEAD
It is not uncommon that we have our data spread out in different tables and need to bring those together for analysis. In this example we will combine the numbers of trafficstops for black and white drivers per county together with the numbers of the black and white total population for these counties. The population data are the estimated values of the 5 year average from the 2011-2015 American Community Survey (ACS):
=======
It is not uncommon that we have our data spread out in different tables and need to bring those together for analysis. In this example we will combine the numbers of trafficstops for black and white drivers per county together with the numbers of the black and white total population for these counties. Toe population data are the estimated values of the 5 year average from the 2011-2015 American Community Survey (ACS):
\textgreater\textgreater\textgreater\textgreater\textgreater\textgreater\textgreater{} 1aab4b954ad0f82372da78447cb5bc64fdc759fe

\begin{Shaded}
\begin{Highlighting}[]
\NormalTok{MS_bw_pop <-}\StringTok{ }\KeywordTok{read.csv}\NormalTok{(}\StringTok{"data/MS_acs2015_bw.csv"}\NormalTok{)}
\KeywordTok{head}\NormalTok{(MS_bw_pop)}
\end{Highlighting}
\end{Shaded}

\begin{verbatim}
#>              County  FIPS black_pop white_pop bw_pop
#> 1      Jones County 28067     19711     47154  66865
#> 2 Lauderdale County 28075     33893     43482  77375
#> 3       Pike County 28113     21028     18282  39310
#> 4    Hancock County 28045      4172     39686  43858
#> 5     Holmes County 28051     15498      3105  18603
#> 6    Jackson County 28059     30704    101686 132390
\end{verbatim}

In a first step we will use a prevous \texttt{dplyr} command to count all the trafficstops per county.

\begin{Shaded}
\begin{Highlighting}[]
\NormalTok{trafficstops }\OperatorTok\StringTok{ }
\StringTok{  }\KeywordTok{group_by}\NormalTok{(county_name) }\OperatorTok\StringTok{ }
\StringTok{  }\KeywordTok{summarise}\NormalTok{(}\DataTypeTok{n_stops =} \KeywordTok{n}\NormalTok{())}
\end{Highlighting}
\end{Shaded}

\begin{verbatim}
#> # A tibble: 82 x 2
#>    county_name      n_stops
#>    <fct>              <int>
#>  1 Adams County         942
#>  2 Alcorn County       3345
#>  3 Amite County        2921
#>  4 Attala County       4203
#>  5 Benton County        214
#>  6 Bolivar County      4526
#>  7 Calhoun County      1658
#>  8 Carroll County      1788
#>  9 Chickasaw County    3869
#> 10 Choctaw County       613
#> # ... with 72 more rows
\end{verbatim}

We will then pipe this into our next operation where we bring the two tables together. We will use \texttt{left\_join}, which returns all rows from the left table, and all columns from the left and the right table. As unique ID, which uniquely identifies the corresponding records in each table we use the County Names.

\begin{Shaded}
\begin{Highlighting}[]
\NormalTok{trafficstops }\OperatorTok\StringTok{ }
\StringTok{  }\KeywordTok{group_by}\NormalTok{(county_name) }\OperatorTok\StringTok{ }
\StringTok{  }\KeywordTok{summarise}\NormalTok{(}\DataTypeTok{n_stops =} \KeywordTok{n}\NormalTok{()) }\OperatorTok\StringTok{ }
\StringTok{  }\KeywordTok{left_join}\NormalTok{(MS_bw_pop, }\DataTypeTok{by =} \KeywordTok{c}\NormalTok{(}\StringTok{"county_name"}\NormalTok{ =}\StringTok{ "County"}\NormalTok{)) }\OperatorTok\StringTok{ }
\StringTok{  }\KeywordTok{head}\NormalTok{()}
\end{Highlighting}
\end{Shaded}

\begin{verbatim}
#> # A tibble: 6 x 6
#>   county_name    n_stops  FIPS black_pop white_pop bw_pop
#>   <fct>            <int> <int>     <int>     <int>  <int>
#> 1 Adams County       942 28001     17757     12856  30613
#> 2 Alcorn County     3345 28003      4281     31563  35844
#> 3 Amite County      2921 28005      5416      7395  12811
#> 4 Attala County     4203 28007      8194     10649  18843
#> 5 Benton County      214 28009      3078      5166   8244
#> 6 Bolivar County    4526 28011     21648     11197  32845
\end{verbatim}

Now we can, for example calculate the percentage of the population that gets stopped in each county.

\begin{quote}
Challenge

Which county has the highest (lowest) percentage of stopped drivers?
Use the snippet from above and pipe into the additional operations
to do this.
\end{quote}

\texttt{dplyr} join functions are generally equivalent \texttt{merge} from the base command, but there are a few advantages:

\begin{itemize}
\tightlist
\item
  rows are kept in existing order
\item
  much faster
\item
  tells you what keys you're merging by (if you don't supply)
\item
  also work with database tables.
\end{itemize}

\url{https://groups.google.com/d/msg/manipulatr/OuAPC4VyfIc/Qnt8mDfq0WwJ}

See \texttt{?dplyr::join} for all the possible joins.

\hypertarget{tidyr}{%
\chapter{\texorpdfstring{Data Manipulation using \textbf{\texttt{tidyr}}}{Data Manipulation using tidyr}}\label{tidyr}}

\begin{quote}
Learning Objectives

\begin{itemize}
\tightlist
\item
  Understand the concept of a wide and a long table format and for which purpose those formats are useful.
\item
  Understand what key-value pairs are.
\item
  Reshape a data frame from long to wide format and back with the \texttt{pivot\_wider} and \texttt{pivot\_longer} commands from the \textbf{\texttt{tidyr}} package.
\item
  Export a data frame to a .csv file.
\end{itemize}
\end{quote}

\begin{center}\rule{0.5\linewidth}{\linethickness}\end{center}

\texttt{dplyr} pairs nicely with \textbf{\texttt{tidyr}} which enables you to swiftly convert between different data formats for plotting and analysis.

The package \textbf{\texttt{tidyr}} addresses the common problem of wanting to reshape your data for plotting and use by different R functions. Sometimes we want data sets where we have one row per observation. Sometimes we want a data frame where each observation type has its own column, and rows are instead more aggregated groups - like surveys, where each column represents an answer. Moving back and forth between these formats is nontrivial, and \textbf{\texttt{tidyr}} gives you tools for this and more sophisticated data manipulation.

To learn more about \textbf{\texttt{tidyr}} after the workshop, you may want to check out this \href{https://github.com/rstudio/cheatsheets/raw/master/data-import.pdf}{cheatsheet about \textbf{\texttt{tidyr}}}.

\hypertarget{about-long-and-wide-table-format}{%
\section{About long and wide table format}\label{about-long-and-wide-table-format}}

The `long' format is where:

\begin{itemize}
\tightlist
\item
  each column is a variable
\item
  each row is an observation
\end{itemize}

In the `long' format, you usually have 1 column for the observed variable and
the other columns are ID variables.

For the `wide' format a row, for example could be a reserach subject for which you have multiple observation variables containing the same type of data, for example responses to a set of survey questions, or repeated observations over time, or a mix of both. Here is an example:

\begin{table}[ht]
\centering
\begin{tabular}{rlrrr}
  \hline
 & subject\_ID & question\_1 & question\_2 & question\_3 \\ 
  \hline
1 & A & 4.00 & 3.00 & 4.00 \\ 
  2 & B & 4.00 & 1.00 & 5.00 \\ 
  3 & C & 2.00 & 5.00 & 2.00 \\ 
   \hline
\end{tabular}
\end{table}

You may find data input may be simpler or some other
applications may prefer the `wide' format. However, many of \texttt{R}`s functions have
been designed assuming you have 'long' format data. This tutorial will help you
efficiently transform your data regardless of original format.

\begin{figure}
\includegraphics[width=0.3\linewidth]{img/wide-vs-long} \caption{Wide vs. Long Table Format}\label{fig:wide-vs-long}
\end{figure}

The choice of data format affects readability. For humans, the wide format is often more intuitive, since we can often see more of the data on the screen due to its shape. However, the long format is more machine readable and is closer to the formatting of databases. The \texttt{ID} variables in our dataframes are similar to the fields in a database and observed variables are like the database values.

\begin{quote}
Challenge 1

Is trafficstops in a long or wide format?
\end{quote}

\hypertarget{long-to-wide-with-pivot_wider}{%
\section{\texorpdfstring{Long to Wide with \texttt{pivot\_wider}}{Long to Wide with pivot\_wider}}\label{long-to-wide-with-pivot_wider}}

Now let's see this in action. First, using \textbf{\texttt{dplyr}}, let's create a data frame
with the mean age of each driver by gender and county:

\begin{Shaded}
\begin{Highlighting}[]
\NormalTok{trafficstops_ma <-}\StringTok{ }\NormalTok{trafficstops }\OperatorTok
\StringTok{    }\KeywordTok{filter}\NormalTok{(}\OperatorTok{!}\KeywordTok{is.na}\NormalTok{(driver_gender)) }\OperatorTok
\StringTok{    }\KeywordTok{group_by}\NormalTok{(county_name, driver_gender) }\OperatorTok
\StringTok{    }\KeywordTok{summarize}\NormalTok{(}\DataTypeTok{mean_age =} \KeywordTok{mean}\NormalTok{(driver_age, }\DataTypeTok{na.rm =} \OtherTok{TRUE}\NormalTok{))}

\NormalTok{trafficstops_ma}
\end{Highlighting}
\end{Shaded}

\begin{verbatim}
#> # A tibble: 164 x 3
#> # Groups:   county_name [82]
#>    county_name   driver_gender mean_age
#>    <fct>         <fct>            <dbl>
#>  1 Adams County  female            36.7
#>  2 Adams County  male              38.4
#>  3 Alcorn County female            33.3
#>  4 Alcorn County male              34.1
#>  5 Amite County  female            38.3
#>  6 Amite County  male              40.3
#>  7 Attala County female            36.7
#>  8 Attala County male              38.1
#>  9 Benton County female            32.1
#> 10 Benton County male              34.4
#> # ... with 154 more rows
\end{verbatim}

Now, to make this long data wide, we use \texttt{pivot\_wider} from \texttt{tidyr} to turn the driver gender into columns. In addition to our data table we provide \texttt{pivot\_wider} with two arguments: \texttt{names\_from} describes which column to use for name of the output column, and \texttt{values\_from} tells it from column to get the cell values. We'll use a pipe so we can ignore the data argument.

\begin{Shaded}
\begin{Highlighting}[]
\NormalTok{trafficstops_ma_wide <-}\StringTok{ }\NormalTok{trafficstops_ma }\OperatorTok
\StringTok{  }\KeywordTok{pivot_wider}\NormalTok{(}\DataTypeTok{names_from =}\NormalTok{ driver_gender, }
              \DataTypeTok{values_from =}\NormalTok{ mean_age) }

\NormalTok{trafficstops_ma_wide}
\end{Highlighting}
\end{Shaded}

\begin{verbatim}
#> # A tibble: 82 x 3
#> # Groups:   county_name [82]
#>    county_name      female  male
#>    <fct>             <dbl> <dbl>
#>  1 Adams County       36.7  38.4
#>  2 Alcorn County      33.3  34.1
#>  3 Amite County       38.3  40.3
#>  4 Attala County      36.7  38.1
#>  5 Benton County      32.1  34.4
#>  6 Bolivar County     33.2  36.3
#>  7 Calhoun County     33.3  34.8
#>  8 Carroll County     33.8  36.0
#>  9 Chickasaw County   33.2  34.6
#> 10 Choctaw County     35.8  36.8
#> # ... with 72 more rows
\end{verbatim}

We can now do things like compare the mean age of men against women drivers. As example we use the age difference to find the counties with the largest and with the smallest number. (A negative number means that female drivers are on average older than male drivers, a positive number means that male drivers are on average older than women drivers.)

\begin{Shaded}
\begin{Highlighting}[]
\NormalTok{trafficstops_ma_wide }\OperatorTok\StringTok{ }
\StringTok{  }\KeywordTok{mutate}\NormalTok{(}\DataTypeTok{agediff =}\NormalTok{ male }\OperatorTok{-}\StringTok{ }\NormalTok{female) }\OperatorTok\StringTok{ }
\StringTok{  }\KeywordTok{ungroup}\NormalTok{() }\OperatorTok
\StringTok{  }\KeywordTok{filter}\NormalTok{(agediff }\OperatorTok\StringTok{ }\KeywordTok{range}\NormalTok{(agediff))}
\end{Highlighting}
\end{Shaded}

\begin{verbatim}
#> # A tibble: 2 x 4
#>   county_name      female  male agediff
#>   <fct>             <dbl> <dbl>   <dbl>
#> 1 Neshoba County     35.1  31.1   -3.94
#> 2 Yalobusha County   33.4  39.4    5.99
\end{verbatim}

Note that \texttt{trafficstops\_ma\_wide} is derived from \texttt{trafficstops\_ma}, and is a ``grouped'' data frame, which was created with the \texttt{group\_by} function above. (Check \texttt{class(trafficstops\_ma)} and \texttt{class(trafficstops\_ma\_wide)}). That means that any instruction that follows will operate on each group (in this case county) separately. That may be ok for some instances (like \texttt{mutate}), but if we are interested in retrieving the maximum and the minumim age difference over \emph{all} counties we need to \texttt{ungroup} the tibble to have the \texttt{filter} command operate on the entire dataset instead of each group (i.e.~county).

\hypertarget{wide-to-long-with-pivot_longer}{%
\section{\texorpdfstring{Wide to long with \texttt{pivot\_longer}}{Wide to long with pivot\_longer}}\label{wide-to-long-with-pivot_longer}}

What if we had the opposite problem, and wanted to go from a wide to long
format? For that, we use \texttt{pivot\_longer}, which will increase the number of rows and decrease the number of columns. We provide the functino with thee arguments: \texttt{cols} which are the columns we want to pivot into the long format, \texttt{names\_to}, which is a string specifying the name of the column to create from the data stored in the column names, and \texttt{values\_to}, which is also a string, specifying the name of the column to create from the data stored in cell values.
So, to go backwards from \texttt{trafficstops\_ma\_wide}, and exclude \texttt{county\_name} from the long, we would do the following:

\begin{Shaded}
\begin{Highlighting}[]
\NormalTok{trafficstops_ma_long <-}\StringTok{ }\NormalTok{trafficstops_ma_wide }\OperatorTok
\StringTok{  }\KeywordTok{pivot_longer}\NormalTok{(}\DataTypeTok{cols =} \OperatorTok{-}\NormalTok{county_name, }
               \DataTypeTok{names_to =} \StringTok{"driver_gender"}\NormalTok{, }\CommentTok{# name is a string!}
               \DataTypeTok{values_to =} \StringTok{"mean_age"}\NormalTok{)     }\CommentTok{# also a string}

\NormalTok{trafficstops_ma_long}
\end{Highlighting}
\end{Shaded}

\begin{verbatim}
#> # A tibble: 164 x 3
#> # Groups:   county_name [82]
#>    county_name   driver_gender mean_age
#>    <fct>         <chr>            <dbl>
#>  1 Adams County  female            36.7
#>  2 Adams County  male              38.4
#>  3 Alcorn County female            33.3
#>  4 Alcorn County male              34.1
#>  5 Amite County  female            38.3
#>  6 Amite County  male              40.3
#>  7 Attala County female            36.7
#>  8 Attala County male              38.1
#>  9 Benton County female            32.1
#> 10 Benton County male              34.4
#> # ... with 154 more rows
\end{verbatim}

We could also have used a specification for what columns to include. This can be
useful if you have a large number of identifying columns, and it's easier to
specify what to gather than what to leave alone. And if the columns are adjacent to each other, we don't even need to list them all out -- we can use the \texttt{:} operator!

\begin{Shaded}
\begin{Highlighting}[]
\NormalTok{trafficstops_ma_wide }\OperatorTok
\StringTok{  }\KeywordTok{pivot_longer}\NormalTok{(}\DataTypeTok{cols =}\NormalTok{ male}\OperatorTok{:}\NormalTok{female,      }\CommentTok{# this also works}
               \DataTypeTok{names_to =} \StringTok{"driver_gender"}\NormalTok{, }
               \DataTypeTok{values_to =} \StringTok{"mean_age"}\NormalTok{)}
\end{Highlighting}
\end{Shaded}

\begin{verbatim}
#> # A tibble: 164 x 3
#> # Groups:   county_name [82]
#>    county_name   driver_gender mean_age
#>    <fct>         <chr>            <dbl>
#>  1 Adams County  male              38.4
#>  2 Adams County  female            36.7
#>  3 Alcorn County male              34.1
#>  4 Alcorn County female            33.3
#>  5 Amite County  male              40.3
#>  6 Amite County  female            38.3
#>  7 Attala County male              38.1
#>  8 Attala County female            36.7
#>  9 Benton County male              34.4
#> 10 Benton County female            32.1
#> # ... with 154 more rows
\end{verbatim}

There are many powerful operations you can do with the \texttt{pivot\_*} functions. To learn more review the vignette:

\begin{Shaded}
\begin{Highlighting}[]
\KeywordTok{vignette}\NormalTok{(}\StringTok{"pivot"}\NormalTok{)}
\end{Highlighting}
\end{Shaded}

\begin{quote}
Challenge

1.From the trafficstops dataframe create a wide data frame \texttt{tr\_wide} with
``year'' as columns, each row is a different violation (``violation\_raw''),
and the values are the
number of traffic stops per each violation, roughly like this:

\texttt{violation\_raw\ \ \textbar{}\ 2013\ \textbar{}\ 2014\ \textbar{}\ 2015\ ...}
\texttt{Improper\ turn\ \ \textbar{}\ \ \ 65\ \textbar{}\ \ \ \ 54\textbar{}\ \ \ 67\ ...}
\texttt{Speeding\ \ \ \ \ \ \ \textbar{}\ \ 713\ \textbar{}\ \ \ 948\textbar{}\ \ 978\ ...}
\texttt{...}

Use \texttt{year()} from the lubridate package. Hint: You will need to summarize
and count the traffic stops before reshaping the table.
\end{quote}

\begin{quote}
\begin{enumerate}
\def\labelenumi{\arabic{enumi}.}
\setcounter{enumi}{1}
\tightlist
\item
  Now take the data frame, and make it long again, so each row is a
  unique violation\_raw - year combination, like this:
\end{enumerate}

\texttt{violation\_raw\ \textbar{}\ year\ \textbar{}\ n\ of\ stops}
\texttt{Improper\ turn\ \textbar{}\ 2013\ \textbar{}\ 65}
\texttt{Improper\ turn\ \textbar{}\ 2014\ \textbar{}\ 54}
\texttt{...\ etc}
\end{quote}

\hypertarget{exporting-data}{%
\section{Exporting data}\label{exporting-data}}

Similar to the \texttt{read.csv()} function used for reading CSV files into R, there is a \texttt{write.csv()} function that generates CSV files from data frames.

Before using \texttt{write.csv()}, we are going to create a new folder, \texttt{data\_output},
in our working directory that will store this generated dataset. We don't want
to write generated datasets in the same directory as our raw data. It's good
practice to keep them separate. The \texttt{data} folder should only contain the raw,
unaltered data, and should be left alone to make sure we don't delete or modify
it. In contrast, our script will generate the contents of the \texttt{data\_output}
directory, so even if the files it contains are deleted, we can always
re-generate them.

We can save the table generated above in our \texttt{data\_output}
folder. By default, \texttt{write.csv()} includes a column with row names (in our case
these names are the row numbers), so we need to add \texttt{row.names\ =\ FALSE} so
they are not included:

\begin{Shaded}
\begin{Highlighting}[]
\KeywordTok{write.csv}\NormalTok{(trafficstops_ma_wide, }\StringTok{"data_output/MS_county_stops.csv"}\NormalTok{, }\DataTypeTok{row.names =}\NormalTok{ F)}
\end{Highlighting}
\end{Shaded}

\hypertarget{data-visualization-with-ggplot2}{%
\chapter{\texorpdfstring{Data Visualization with \texttt{ggplot2}}{Data Visualization with ggplot2}}\label{data-visualization-with-ggplot2}}

\begin{quote}
Learning Objectives

\begin{itemize}
\tightlist
\item
  Bind a data frame to a plot
\item
  Select variables to be plotted and variables to define the presentation such as size, shape, color, transparency, etc. by defining aesthetics (\texttt{aes})
\item
  Add a graphical representation of the data in the plot (points, lines, bars) adding ``geoms'' layers
\item
  Produce scatter plots, barplots, boxplots, and line plots using ggplot.
\item
  Modify the aesthetics for the entire plot as well as for individual ``geoms'' layers
\item
  Modify plot elements (labels, text, scale, orientation)
\item
  Group observations by a factor variable
\item
  Break up plot into multiple panels (facetting)
\item
  Apply ggplot themes and create and apply customized themes
\item
  Save a plot created by ggplot as an image
\end{itemize}
\end{quote}

\begin{center}\rule{0.5\linewidth}{\linethickness}\end{center}

We start by loading the required packages. \textbf{\texttt{ggplot2}} is included in the \textbf{\texttt{tidyverse}} package.

\begin{Shaded}
\begin{Highlighting}[]
\KeywordTok{library}\NormalTok{(tidyverse)}
\end{Highlighting}
\end{Shaded}

First, load the data we saved in the previous lesson.

\begin{Shaded}
\begin{Highlighting}[]
\NormalTok{MS_county_stops <-}\StringTok{ }\KeywordTok{read.csv}\NormalTok{(}\StringTok{'data_output/MS_county_stops.csv'}\NormalTok{)}
\end{Highlighting}
\end{Shaded}

(If you need to, you can also download the data from here: \url{http://bit.ly/MS_county_stops}. Create a folder \texttt{data\_output} in your working directory then download with \texttt{download.file(\textquotesingle{}http://bit.ly/MS\_county\_stops\textquotesingle{},\ \textquotesingle{}data\_output/MS\_county\_stops.csv})

\hypertarget{plotting-with-ggplot2}{%
\section{\texorpdfstring{Plotting with \textbf{\texttt{ggplot2}}}{Plotting with ggplot2}}\label{plotting-with-ggplot2}}

\textbf{\texttt{ggplot2}} is a plotting package that makes it simple to create complex plots
from data in a data frame. It provides a more programmatic interface for
specifying what variables to plot, how they are displayed, and general visual
properties, so we only need minimal changes if the underlying data change or if
we decide to change from a bar plot to a scatterplot. This helps in creating
publication quality plots with minimal amounts of adjustments and tweaking.

ggplot graphics are built step by step by adding new elements using the \texttt{+} sign.

To build a ggplot we need to:

\begin{itemize}
\tightlist
\item
  bind the plot to a specific data frame using the \texttt{data} argument
\end{itemize}

\begin{Shaded}
\begin{Highlighting}[]
\KeywordTok{ggplot}\NormalTok{(}\DataTypeTok{data =}\NormalTok{ MS_county_stops)}
\end{Highlighting}
\end{Shaded}

\begin{itemize}
\tightlist
\item
  define aesthetics (\texttt{aes}), by selecting the variables to be plotted and the variables to define the presentation such as plotting size, shape color, etc.
\end{itemize}

\begin{Shaded}
\begin{Highlighting}[]
\KeywordTok{ggplot}\NormalTok{(}\DataTypeTok{data =}\NormalTok{ MS_county_stops, }\KeywordTok{aes}\NormalTok{(}\DataTypeTok{x =}\NormalTok{ female, }\DataTypeTok{y =}\NormalTok{ male))}
\end{Highlighting}
\end{Shaded}

\begin{itemize}
\tightlist
\item
  add ``geoms'' -- a graphical representation of the data in the plot (points, lines, bars). To add a geom to the plot use \texttt{+} operator
\end{itemize}

\begin{Shaded}
\begin{Highlighting}[]
\KeywordTok{ggplot}\NormalTok{(}\DataTypeTok{data =}\NormalTok{ MS_county_stops, }\KeywordTok{aes}\NormalTok{(}\DataTypeTok{x =}\NormalTok{ female, }\DataTypeTok{y =}\NormalTok{ male)) }\OperatorTok{+}
\StringTok{  }\KeywordTok{geom_point}\NormalTok{()}
\end{Highlighting}
\end{Shaded}

\includegraphics[width=0.7\linewidth]{R-data-wrangling_files/figure-latex/first-ggplot-1}

The \texttt{+} in the \textbf{\texttt{ggplot2}} package is particularly useful because it allows you
to modify existing \texttt{ggplot} objects. This means you can easily set up plot
``templates'' and conveniently explore different types of plots, so the above
plot can also be generated with code like this:

\begin{Shaded}
\begin{Highlighting}[]
\CommentTok{# Assign plot to a variable}
\NormalTok{MS_plot <-}\StringTok{ }\KeywordTok{ggplot}\NormalTok{(}\DataTypeTok{data =}\NormalTok{ MS_county_stops, }\KeywordTok{aes}\NormalTok{(}\DataTypeTok{x =}\NormalTok{ female, }\DataTypeTok{y =}\NormalTok{ male))}

\CommentTok{# Draw the plot}
\NormalTok{MS_plot }\OperatorTok{+}\StringTok{ }\KeywordTok{geom_point}\NormalTok{()}
\end{Highlighting}
\end{Shaded}

Also, conveniently, \texttt{ggplot} works with pipes. The code below does the same thing as above:

\begin{Shaded}
\begin{Highlighting}[]
\NormalTok{MS_county_stops }\OperatorTok\StringTok{ }
\StringTok{  }\KeywordTok{ggplot}\NormalTok{(}\KeywordTok{aes}\NormalTok{(}\DataTypeTok{x =}\NormalTok{ female, }\DataTypeTok{y =}\NormalTok{ male)) }\OperatorTok{+}\StringTok{ }
\StringTok{  }\KeywordTok{geom_point}\NormalTok{() }
\end{Highlighting}
\end{Shaded}

We pipe the content of the table into \texttt{ggplot()}, so we can omit the first (\texttt{data\ =} argument).

Notes:

\begin{itemize}
\tightlist
\item
  Any parameters you set in the \texttt{ggplot()} function can be seen by any geom layers
  that you add (i.e., these are universal plot settings). This includes the x and y axis you set up in \texttt{aes()}.
\item
  Any parameters you set in the \texttt{geom\_*()} function are treated independently of (and override) the settings defined globally in the \texttt{ggplot()} function.
\item
  Geoms are plotted in the order they are added after each \texttt{+}, that means geoms last added will display on top of prior geoms.
\item
  The \texttt{+} sign used to add layers \textbf{must be placed at the end of each line} containing
  a layer. If, instead, the \texttt{+} sign is added in the line before the other layer,
  \textbf{\texttt{ggplot2}} will not add the new layer and will return an error message.
\end{itemize}

\begin{Shaded}
\begin{Highlighting}[]
\CommentTok{# this is the correct syntax for adding layers}
\NormalTok{MS_plot }\OperatorTok{+}
\StringTok{  }\KeywordTok{geom_point}\NormalTok{()}

\CommentTok{# this will not add the new layer and will return an error message}
\NormalTok{MS_plot}
  \OperatorTok{+}\StringTok{ }\KeywordTok{geom_point}\NormalTok{()}
\end{Highlighting}
\end{Shaded}

To learn more about \textbf{\texttt{ggplot}} after the workshop, you may want to check out this \href{https://www.rstudio.com/wp-content/uploads/2016/11/ggplot2-cheatsheet-2.1.pdf}{cheatsheet about \textbf{\texttt{ggplot}}}.

\hypertarget{building-your-plots-iteratively}{%
\section{Building your plots iteratively}\label{building-your-plots-iteratively}}

Building plots with ggplot can be of great help when you engage in exploratory data analysis. It is typically an iterative process, where you go back and forth between your data and their graphical representation, which helps you in the process of getting to know your data better.

We can start modifying this plot to extract more information from it. For instance, we can add transparency (\texttt{alpha}) to avoid overplotting:

\begin{Shaded}
\begin{Highlighting}[]
  \KeywordTok{ggplot}\NormalTok{(MS_county_stops, }\KeywordTok{aes}\NormalTok{(}\DataTypeTok{x =}\NormalTok{ female, }\DataTypeTok{y =}\NormalTok{ male)) }\OperatorTok{+}\StringTok{ }
\StringTok{  }\KeywordTok{geom_point}\NormalTok{(}\DataTypeTok{alpha =} \FloatTok{0.3}\NormalTok{)}
\end{Highlighting}
\end{Shaded}

\includegraphics[width=0.7\linewidth]{R-data-wrangling_files/figure-latex/adding-transparency-1}

We can also add a color for all the points:

\begin{Shaded}
\begin{Highlighting}[]
\KeywordTok{ggplot}\NormalTok{(MS_county_stops, }\KeywordTok{aes}\NormalTok{(}\DataTypeTok{x =}\NormalTok{ female, }\DataTypeTok{y =}\NormalTok{ male)) }\OperatorTok{+}\StringTok{ }
\StringTok{  }\KeywordTok{geom_point}\NormalTok{(}\DataTypeTok{alpha =} \FloatTok{0.3}\NormalTok{, }\DataTypeTok{color=} \StringTok{"blue"}\NormalTok{)}
\end{Highlighting}
\end{Shaded}

\includegraphics[width=0.7\linewidth]{R-data-wrangling_files/figure-latex/adding-color-1}

We can add another layer to the plot with \texttt{+}:

\begin{Shaded}
\begin{Highlighting}[]
\KeywordTok{ggplot}\NormalTok{(MS_county_stops, }\KeywordTok{aes}\NormalTok{(}\DataTypeTok{x =}\NormalTok{ female, }\DataTypeTok{y =}\NormalTok{ male)) }\OperatorTok{+}\StringTok{ }
\StringTok{  }\KeywordTok{geom_point}\NormalTok{(}\DataTypeTok{alpha =} \FloatTok{0.3}\NormalTok{, }\DataTypeTok{color=} \StringTok{"blue"}\NormalTok{) }\OperatorTok{+}
\StringTok{  }\KeywordTok{geom_abline}\NormalTok{()}
\end{Highlighting}
\end{Shaded}

\includegraphics[width=0.7\linewidth]{R-data-wrangling_files/figure-latex/add-abline-1}

\begin{quote}
Challenge

Modify the plot above to display different color for both points and abline. How might you change the size of the dots?
\end{quote}

\hypertarget{barplot}{%
\section{Barplot}\label{barplot}}

There are two types of bar charts in ggplot, \texttt{geom\_bar} and \texttt{geom\_col}. \texttt{geom\_bar} makes the height of the bar proportional to the number of cases in each group and counts the number of cases at each x position.

Let's go back go our trafficstops dataframe. If we wanted to see how many violations we have of each type could say:

\begin{Shaded}
\begin{Highlighting}[]
\KeywordTok{ggplot}\NormalTok{(trafficstops, }\KeywordTok{aes}\NormalTok{(violation)) }\OperatorTok{+}\StringTok{ }
\StringTok{  }\KeywordTok{geom_bar}\NormalTok{()}
\end{Highlighting}
\end{Shaded}

\includegraphics[width=0.7\linewidth]{R-data-wrangling_files/figure-latex/simple-bar-1}

As we have seen we could color the bars, but instead of \texttt{color} we use \texttt{fill}. (What happens when you use \texttt{color}?)

\begin{Shaded}
\begin{Highlighting}[]
\KeywordTok{ggplot}\NormalTok{(trafficstops, }\KeywordTok{aes}\NormalTok{(violation)) }\OperatorTok{+}\StringTok{ }
\StringTok{  }\KeywordTok{geom_bar}\NormalTok{(}\DataTypeTok{fill =} \StringTok{"green"}\NormalTok{)}
\end{Highlighting}
\end{Shaded}

\includegraphics[width=0.7\linewidth]{R-data-wrangling_files/figure-latex/color-bar-simple-1}

Instead of coloring everything the same we could also color by another category, say gender. For this we have to set the parameter within the \texttt{aes()} function, which takes care of mapping the values to different colors:

\begin{Shaded}
\begin{Highlighting}[]
\KeywordTok{ggplot}\NormalTok{(trafficstops, }\KeywordTok{aes}\NormalTok{(violation)) }\OperatorTok{+}\StringTok{ }
\StringTok{  }\KeywordTok{geom_bar}\NormalTok{(}\KeywordTok{aes}\NormalTok{(}\DataTypeTok{fill =}\NormalTok{ driver_gender))}
\end{Highlighting}
\end{Shaded}

\includegraphics[width=0.7\linewidth]{R-data-wrangling_files/figure-latex/color-bar-gender-1}

If we wanted to see the proportions within each category we can tell ggplot to stretch the bars between 0 and 1, we can set the position parameter to `fill':

\begin{Shaded}
\begin{Highlighting}[]
\KeywordTok{ggplot}\NormalTok{(trafficstops, }\KeywordTok{aes}\NormalTok{(violation)) }\OperatorTok{+}\StringTok{ }
\StringTok{  }\KeywordTok{geom_bar}\NormalTok{(}\KeywordTok{aes}\NormalTok{(}\DataTypeTok{fill =}\NormalTok{ driver_gender), }\DataTypeTok{position =} \StringTok{"fill"}\NormalTok{)}
\end{Highlighting}
\end{Shaded}

\includegraphics[width=0.7\linewidth]{R-data-wrangling_files/figure-latex/color-bar-stretch-1}

To make this more readable we add anoter function to flip the coordinates.

\begin{Shaded}
\begin{Highlighting}[]
\KeywordTok{ggplot}\NormalTok{(trafficstops, }\KeywordTok{aes}\NormalTok{(violation)) }\OperatorTok{+}\StringTok{ }
\StringTok{  }\KeywordTok{geom_bar}\NormalTok{(}\KeywordTok{aes}\NormalTok{(}\DataTypeTok{fill =}\NormalTok{ driver_gender), }\DataTypeTok{position =} \StringTok{"fill"}\NormalTok{) }\OperatorTok{+}
\StringTok{  }\KeywordTok{coord_flip}\NormalTok{()}
\end{Highlighting}
\end{Shaded}

\includegraphics[width=0.7\linewidth]{R-data-wrangling_files/figure-latex/color-bar-stretch-flip-1}

The other type of barchart, \texttt{geom\_col}, is used if you want the heights of the bars to represent the actual data values. In this example, we first generate a summary statistics, and then pipe it into ggplot to visualize the values we have calculated.

\begin{Shaded}
\begin{Highlighting}[]
\NormalTok{trafficstops }\OperatorTok\StringTok{ }
\StringTok{    }\KeywordTok{group_by}\NormalTok{(violation) }\OperatorTok\StringTok{ }
\StringTok{    }\KeywordTok{summarize}\NormalTok{(}\DataTypeTok{mean_age =} \KeywordTok{mean}\NormalTok{(driver_age, }\DataTypeTok{na.rm =} \OtherTok{TRUE}\NormalTok{)) }\OperatorTok
\StringTok{    }\KeywordTok{ggplot}\NormalTok{(}\KeywordTok{aes}\NormalTok{(}\DataTypeTok{x =}\NormalTok{ violation, }\DataTypeTok{y =}\NormalTok{ mean_age)) }\OperatorTok{+}\StringTok{ }\CommentTok{# reorder}
\StringTok{    }\KeywordTok{geom_col}\NormalTok{() }\OperatorTok{+}\StringTok{ }
\StringTok{    }\KeywordTok{coord_flip}\NormalTok{() }\CommentTok{# flip the two axes}
\end{Highlighting}
\end{Shaded}

\includegraphics{R-data-wrangling_files/figure-latex/demograph-colplot-1.pdf}

We can also reorder the barplots, so we can see them in descending order.

\begin{Shaded}
\begin{Highlighting}[]
\NormalTok{trafficstops }\OperatorTok\StringTok{ }
\StringTok{  }\KeywordTok{group_by}\NormalTok{(violation) }\OperatorTok\StringTok{ }
\StringTok{  }\KeywordTok{summarize}\NormalTok{(}\DataTypeTok{mean_age =} \KeywordTok{mean}\NormalTok{(driver_age, }\DataTypeTok{na.rm =} \OtherTok{TRUE}\NormalTok{)) }\OperatorTok
\StringTok{  }\KeywordTok{ggplot}\NormalTok{(}\KeywordTok{aes}\NormalTok{(}\DataTypeTok{x =} \KeywordTok{reorder}\NormalTok{(violation, mean_age), }\DataTypeTok{y =}\NormalTok{ mean_age)) }\OperatorTok{+}\StringTok{ }\CommentTok{# reorder}
\StringTok{    }\KeywordTok{geom_col}\NormalTok{() }\OperatorTok{+}\StringTok{ }
\StringTok{    }\KeywordTok{coord_flip}\NormalTok{() }\CommentTok{# flip the two axes}
\end{Highlighting}
\end{Shaded}

\includegraphics{R-data-wrangling_files/figure-latex/demograph-colplot-reorder-1.pdf}

\begin{quote}
Challenge

Make a barplot that shows the proportion of stops per race for within each gender category. How could you get rid of the NAs?
\end{quote}

\hypertarget{boxplot}{%
\section{Boxplot}\label{boxplot}}

For this segment let's extract and work with the stops for Yazoo County only. We will also remove some NAs here.

\begin{Shaded}
\begin{Highlighting}[]
\NormalTok{Yazoo_stops <-}\StringTok{ }\NormalTok{trafficstops }\OperatorTok\StringTok{ }
\StringTok{  }\KeywordTok{filter}\NormalTok{(county_name }\OperatorTok{==}\StringTok{ "Yazoo County"}\NormalTok{,}
         \OperatorTok{!}\KeywordTok{is.na}\NormalTok{(driver_age))}
\end{Highlighting}
\end{Shaded}

We can use boxplots to visualize the distribution of driver age within each violation:

\begin{Shaded}
\begin{Highlighting}[]
\KeywordTok{ggplot}\NormalTok{(Yazoo_stops, }\KeywordTok{aes}\NormalTok{(}\DataTypeTok{x =}\NormalTok{ violation, }\DataTypeTok{y =}\NormalTok{ driver_age)) }\OperatorTok{+}
\StringTok{    }\KeywordTok{geom_boxplot}\NormalTok{()}
\end{Highlighting}
\end{Shaded}

\includegraphics[width=0.7\linewidth]{R-data-wrangling_files/figure-latex/boxplot-1}

By adding points to boxplot, we can have a better idea of the number of
measurements and of their distribution.

\begin{Shaded}
\begin{Highlighting}[]
\KeywordTok{ggplot}\NormalTok{(}\DataTypeTok{data =}\NormalTok{ Yazoo_stops, }\KeywordTok{aes}\NormalTok{(}\DataTypeTok{x =}\NormalTok{ violation, }\DataTypeTok{y =}\NormalTok{ driver_age)) }\OperatorTok{+}
\StringTok{    }\KeywordTok{geom_boxplot}\NormalTok{() }\OperatorTok{+}
\StringTok{    }\KeywordTok{geom_jitter}\NormalTok{()}
\end{Highlighting}
\end{Shaded}

\includegraphics[width=0.7\linewidth]{R-data-wrangling_files/figure-latex/boxplot-with-jitter-1}

That looks quite messy. Let's clean it up by using the \texttt{alpha} parameter to make the dots more transparent and also change their color:

\begin{Shaded}
\begin{Highlighting}[]
\KeywordTok{ggplot}\NormalTok{(}\DataTypeTok{data =}\NormalTok{ Yazoo_stops, }\KeywordTok{aes}\NormalTok{(}\DataTypeTok{x =}\NormalTok{ violation, }\DataTypeTok{y =}\NormalTok{ driver_age)) }\OperatorTok{+}
\StringTok{    }\KeywordTok{geom_boxplot}\NormalTok{() }\OperatorTok{+}
\StringTok{    }\KeywordTok{geom_jitter}\NormalTok{(}\DataTypeTok{alpha =} \FloatTok{0.5}\NormalTok{, }\DataTypeTok{color =} \StringTok{"tomato"}\NormalTok{)}
\end{Highlighting}
\end{Shaded}

\includegraphics[width=0.7\linewidth]{R-data-wrangling_files/figure-latex/boxplot-with-jitter-transparent-1}

Notice how the boxplot layer is behind the jitter layer. We will change the plotting order to keep the boxplot visible.

\begin{Shaded}
\begin{Highlighting}[]
\KeywordTok{ggplot}\NormalTok{(}\DataTypeTok{data =}\NormalTok{ Yazoo_stops, }\KeywordTok{aes}\NormalTok{(}\DataTypeTok{x =}\NormalTok{ violation, }\DataTypeTok{y =}\NormalTok{ driver_age)) }\OperatorTok{+}
\StringTok{    }\KeywordTok{geom_jitter}\NormalTok{(}\DataTypeTok{alpha =} \FloatTok{0.1}\NormalTok{, }\DataTypeTok{color =} \StringTok{"tomato"}\NormalTok{) }\OperatorTok{+}\StringTok{ }
\StringTok{    }\KeywordTok{geom_boxplot}\NormalTok{()}
\end{Highlighting}
\end{Shaded}

\includegraphics[width=0.7\linewidth]{R-data-wrangling_files/figure-latex/boxplot-with-jitter-reordered-1}

And finally we will change the transparency of the box plot so it does not cover the points:

\begin{Shaded}
\begin{Highlighting}[]
\KeywordTok{ggplot}\NormalTok{(}\DataTypeTok{data =}\NormalTok{ Yazoo_stops, }\KeywordTok{aes}\NormalTok{(}\DataTypeTok{x =}\NormalTok{ violation, }\DataTypeTok{y =}\NormalTok{ driver_age)) }\OperatorTok{+}
\StringTok{    }\KeywordTok{geom_jitter}\NormalTok{(}\DataTypeTok{alpha =} \FloatTok{0.1}\NormalTok{, }\DataTypeTok{color =} \StringTok{"tomato"}\NormalTok{) }\OperatorTok{+}
\StringTok{    }\KeywordTok{geom_boxplot}\NormalTok{(}\DataTypeTok{alpha =} \DecValTok{0}\NormalTok{)  }
\end{Highlighting}
\end{Shaded}

\includegraphics[width=0.7\linewidth]{R-data-wrangling_files/figure-latex/boxplot-with-jitter-clean-1}

\begin{quote}
Challenge

Boxplots are useful summaries, but hide the \emph{shape} of the distribution. For
example, if there is a bimodal distribution, it would not be observed with a
boxplot. An alternative to the boxplot is the violin plot (sometimes known as a
beanplot), where the shape (of the density of points) is drawn.

\begin{itemize}
\tightlist
\item
  Replace the box plot with a violin plot; see \texttt{geom\_violin()}.
\end{itemize}

So far, we've looked at the distribution of age within violations Try making a
new plot to explore the distribution of age for another variable:

\begin{itemize}
\tightlist
\item
  Create the age box plot for \texttt{driver\_race}. Overlay the boxplot layer on a jitter layer to show actual measurements.
\end{itemize}
\end{quote}

\hypertarget{plotting-time-series-data}{%
\section{Plotting time series data}\label{plotting-time-series-data}}

To make things a little easer we first convert the \texttt{stop\_date} column we plan to use to Date format and use the \texttt{wday} function to extract and add a new column to our \texttt{trafficstops} data called \texttt{wk\_day} with the weekday for each of these dates. For better understanding we will label the weekdays.

\begin{Shaded}
\begin{Highlighting}[]
\KeywordTok{library}\NormalTok{(lubridate)}
\NormalTok{trafficstops <-}\StringTok{ }\NormalTok{trafficstops }\OperatorTok\StringTok{ }
\StringTok{  }\KeywordTok{mutate}\NormalTok{(}\DataTypeTok{stop_date =} \KeywordTok{ymd}\NormalTok{(stop_date),}
         \DataTypeTok{wk_day =} \KeywordTok{wday}\NormalTok{(stop_date, }\DataTypeTok{label =} \OtherTok{TRUE}\NormalTok{))}
\end{Highlighting}
\end{Shaded}

Let's calculate number of violation per weekday. We will use \texttt{count} for this, like this:

\begin{Shaded}
\begin{Highlighting}[]
\NormalTok{trafficstops }\OperatorTok
\StringTok{  }\KeywordTok{count}\NormalTok{(wk_day, violation) }
\end{Highlighting}
\end{Shaded}

\begin{verbatim}
#> # A tibble: 42 x 3
#>    wk_day violation                    n
#>    <ord>  <fct>                    <int>
#>  1 Sun    Breaks-Lights-etc          363
#>  2 Sun    Careless driving           804
#>  3 Sun    License-Permit-Insurance  4177
#>  4 Sun    Other or unknown          1588
#>  5 Sun    Seat belt                 2313
#>  6 Sun    Speeding                 16522
#>  7 Mon    Breaks-Lights-etc          337
#>  8 Mon    Careless driving           830
#>  9 Mon    License-Permit-Insurance  4086
#> 10 Mon    Other or unknown          1572
#> # ... with 32 more rows
\end{verbatim}

We visualize thise data as a line plot (with -- you guessed it -- \texttt{geom\_line()}) mapping the days to the x axis and counts to the y axis. So we pipe the output from above into ggplot like this:

\begin{Shaded}
\begin{Highlighting}[]
\NormalTok{trafficstops }\OperatorTok
\StringTok{  }\KeywordTok{count}\NormalTok{(wk_day, violation) }\OperatorTok\StringTok{ }
\StringTok{  }\KeywordTok{ggplot}\NormalTok{(}\KeywordTok{aes}\NormalTok{(wk_day, n)) }\OperatorTok{+}
\StringTok{    }\KeywordTok{geom_line}\NormalTok{()}
\end{Highlighting}
\end{Shaded}

\includegraphics{R-data-wrangling_files/figure-latex/first-time-series-1.pdf}

Oops. What happende here? We plotted data for all the violations together! So what ggplot displays is the range of all values for each year in a vertial line. We need to tell ggplot to draw a separate line for each violation by modifying the aesthetic function to include \texttt{group\ =\ violation}:

\begin{Shaded}
\begin{Highlighting}[]
\NormalTok{trafficstops }\OperatorTok
\StringTok{  }\KeywordTok{count}\NormalTok{(wk_day, violation) }\OperatorTok\StringTok{ }
\StringTok{  }\KeywordTok{ggplot}\NormalTok{(}\KeywordTok{aes}\NormalTok{(wk_day, n, }\DataTypeTok{group =}\NormalTok{ violation)) }\OperatorTok{+}
\StringTok{    }\KeywordTok{geom_line}\NormalTok{()}
\end{Highlighting}
\end{Shaded}

\includegraphics[width=0.7\linewidth]{R-data-wrangling_files/figure-latex/time-series-by-violation-1}

We will be able to distinguish violations in the plot if we add colors. (Colors groups automatically if the variable is numeric).

\begin{Shaded}
\begin{Highlighting}[]
\NormalTok{trafficstops }\OperatorTok
\StringTok{  }\KeywordTok{count}\NormalTok{(wk_day, violation) }\OperatorTok\StringTok{ }
\StringTok{  }\KeywordTok{ggplot}\NormalTok{(}\KeywordTok{aes}\NormalTok{(wk_day, n, }\DataTypeTok{group =}\NormalTok{ violation, }\DataTypeTok{color =}\NormalTok{ violation)) }\OperatorTok{+}
\StringTok{    }\KeywordTok{geom_line}\NormalTok{()}
\end{Highlighting}
\end{Shaded}

\includegraphics[width=0.7\linewidth]{R-data-wrangling_files/figure-latex/time-series-with-colors-1}

\hypertarget{faceting}{%
\section{Faceting}\label{faceting}}

ggplot has a special technique called \emph{faceting} that allows to split one plot
into multiple plots based on a factor included in the dataset. So instead of using coloring to separate out different violations, we can make a time series plot for each violation:

\begin{Shaded}
\begin{Highlighting}[]
\NormalTok{trafficstops }\OperatorTok
\StringTok{  }\KeywordTok{count}\NormalTok{(wk_day, violation) }\OperatorTok\StringTok{ }
\StringTok{  }\KeywordTok{ggplot}\NormalTok{(}\KeywordTok{aes}\NormalTok{(wk_day, n, }\DataTypeTok{group =}\NormalTok{ violation)) }\OperatorTok{+}
\StringTok{    }\KeywordTok{geom_line}\NormalTok{() }\OperatorTok{+}
\StringTok{     }\KeywordTok{facet_wrap}\NormalTok{(}\OperatorTok{~}\StringTok{ }\NormalTok{violation)}
\end{Highlighting}
\end{Shaded}

\includegraphics[width=0.7\linewidth]{R-data-wrangling_files/figure-latex/first-facet-1}

Now we would like to split the line in each plot by the race of the driver. To do that we count in the data frame grouped by \texttt{wk\_day}, \texttt{violation}, \textbf{and} \texttt{driver\_race}. We then create the faceted plot by splitting further by race using \texttt{color} and \texttt{group} (within a single plot):

\begin{Shaded}
\begin{Highlighting}[]
\NormalTok{trafficstops }\OperatorTok
\StringTok{  }\KeywordTok{count}\NormalTok{(wk_day, violation, driver_race) }\OperatorTok\StringTok{ }\CommentTok{# add driver race here}
\StringTok{  }\KeywordTok{ggplot}\NormalTok{(}\KeywordTok{aes}\NormalTok{(wk_day, n, }\DataTypeTok{color =}\NormalTok{ driver_race, }\DataTypeTok{group =}\NormalTok{ driver_race)) }\OperatorTok{+}
\StringTok{    }\KeywordTok{geom_line}\NormalTok{() }\OperatorTok{+}
\StringTok{     }\KeywordTok{facet_wrap}\NormalTok{(}\OperatorTok{~}\StringTok{ }\NormalTok{violation)}
\end{Highlighting}
\end{Shaded}

\begin{verbatim}
#> Warning: Factor `driver_race` contains implicit NA, consider using
#> `forcats::fct_explicit_na`
\end{verbatim}

\includegraphics{R-data-wrangling_files/figure-latex/facet-by-violation-and-race-1.pdf}

Note that there is an alternative, the \texttt{facet\_grid} geometry, which allows you to explicitly specify how you want your plots to be
arranged via formula notation (\texttt{rows\ \textasciitilde{}\ columns}; a \texttt{.} can be used as
a placeholder that indicates only one row or column).

\begin{quote}
Challenge

Use what you just learned to create a plot that depicts how the average age
of each driver for the two recorded ethnicities changes through the week.
Hint: make sure you remove the records with driver\_age under 16.
How would you go about visualizing both lines and points on the plot?
How would you split your plot into one per each violation type?
\end{quote}

\hypertarget{ggplot2-themes}{%
\section{\texorpdfstring{\textbf{\texttt{ggplot2}} themes}{ggplot2 themes}}\label{ggplot2-themes}}

\textbf{\texttt{ggplot2}}
comes with several other themes which can be useful to quickly change the look of your visualization. Before we do that we will assign our plot above to a variable.

\begin{Shaded}
\begin{Highlighting}[]
\NormalTok{stops_facet_plot <-}\StringTok{ }\NormalTok{trafficstops }\OperatorTok
\StringTok{  }\KeywordTok{count}\NormalTok{(wk_day, violation, driver_race) }\OperatorTok\StringTok{ }\CommentTok{# add driver race here}
\StringTok{  }\KeywordTok{ggplot}\NormalTok{(}\KeywordTok{aes}\NormalTok{(wk_day, n, }\DataTypeTok{color =}\NormalTok{ driver_race, }\DataTypeTok{group =}\NormalTok{ driver_race)) }\OperatorTok{+}
\StringTok{    }\KeywordTok{geom_line}\NormalTok{() }\OperatorTok{+}
\StringTok{     }\KeywordTok{facet_wrap}\NormalTok{(}\OperatorTok{~}\StringTok{ }\NormalTok{violation)}
\end{Highlighting}
\end{Shaded}

\begin{verbatim}
#> Warning: Factor `driver_race` contains implicit NA, consider using
#> `forcats::fct_explicit_na`
\end{verbatim}

Now we can add another theme, \texttt{theme\_bw()} to change the plot background to white:

\begin{Shaded}
\begin{Highlighting}[]
\NormalTok{stops_facet_plot }\OperatorTok{+}
\StringTok{     }\KeywordTok{theme_bw}\NormalTok{()}
\end{Highlighting}
\end{Shaded}

\includegraphics{R-data-wrangling_files/figure-latex/facet-theming-1.pdf}

The complete list of themes is available
at \url{http://docs.ggplot2.org/current/ggtheme.html}. \texttt{theme\_minimal()} and \texttt{theme\_light()} are popular, and \texttt{theme\_void()} can be useful as a starting point to create a new hand-crafted theme.

The \href{https://cran.r-project.org/web/packages/ggthemes/vignettes/ggthemes.html}{ggthemes} package
provides a wide variety of options (including an Excel 2003 theme).
The \href{https://www.ggplot2-exts.org}{\textbf{\texttt{ggplot2}} extensions website} provides a list
of packages that extend the capabilities of \textbf{\texttt{ggplot2}}, including additional themes.

\hypertarget{customization}{%
\section{Customization}\label{customization}}

There are endless possibilities to customize your plot, particularly when you are ready for publication or presentation. For example, let's change names of axes to something more informative than `wk\_day' and `n' and add a title to the figure:

\begin{Shaded}
\begin{Highlighting}[]
\NormalTok{stops_facet_plot }\OperatorTok{+}
\StringTok{  }\KeywordTok{labs}\NormalTok{(}\DataTypeTok{title =} \StringTok{'Observed violations per day of week'}\NormalTok{,}
         \DataTypeTok{x =} \StringTok{'Weekday of observation'}\NormalTok{,}
         \DataTypeTok{y =} \StringTok{'Number of violations'}\NormalTok{) }\OperatorTok{+}
\StringTok{  }\KeywordTok{theme_bw}\NormalTok{()}
\end{Highlighting}
\end{Shaded}

\includegraphics{R-data-wrangling_files/figure-latex/improved-labels-1.pdf}

The axes have more informative names, but their readability can be improved by
increasing the font size:

\begin{Shaded}
\begin{Highlighting}[]
\NormalTok{stops_facet_plot }\OperatorTok{+}
\StringTok{  }\KeywordTok{labs}\NormalTok{(}\DataTypeTok{title =} \StringTok{'Observed violations per day of week'}\NormalTok{,}
         \DataTypeTok{x =} \StringTok{'Weekday of observation'}\NormalTok{,}
         \DataTypeTok{y =} \StringTok{'Number of violations'}\NormalTok{) }\OperatorTok{+}
\StringTok{  }\KeywordTok{theme_bw}\NormalTok{() }\OperatorTok{+}\StringTok{ }
\StringTok{  }\KeywordTok{theme}\NormalTok{(}\DataTypeTok{text =} \KeywordTok{element_text}\NormalTok{(}\DataTypeTok{size=}\DecValTok{16}\NormalTok{))}
\end{Highlighting}
\end{Shaded}

\includegraphics{R-data-wrangling_files/figure-latex/improved-font-size-1.pdf}

After our manipulations, you may notice that the values on the x-axis are still not properly readable. Let's change the orientation of the labels and adjust them vertically and horizontally so they don't overlap. You can use a 90 degree angle, or experiment to find the appropriate angle for diagonally oriented labels:

\begin{Shaded}
\begin{Highlighting}[]
\NormalTok{stops_facet_plot }\OperatorTok{+}
\StringTok{  }\KeywordTok{labs}\NormalTok{(}\DataTypeTok{title =} \StringTok{'Observed violations per day of week'}\NormalTok{,}
         \DataTypeTok{x =} \StringTok{'Weekday of observation'}\NormalTok{,}
         \DataTypeTok{y =} \StringTok{'Number of violations'}\NormalTok{) }\OperatorTok{+}
\StringTok{  }\KeywordTok{theme_bw}\NormalTok{() }\OperatorTok{+}\StringTok{ }
\StringTok{  }\KeywordTok{theme}\NormalTok{(}\DataTypeTok{axis.text.x =} \KeywordTok{element_text}\NormalTok{(}\DataTypeTok{colour=}\StringTok{"grey40"}\NormalTok{, }\DataTypeTok{size=}\DecValTok{12}\NormalTok{, }\DataTypeTok{angle=}\DecValTok{90}\NormalTok{, }\DataTypeTok{hjust=}\NormalTok{.}\DecValTok{5}\NormalTok{, }\DataTypeTok{vjust=}\NormalTok{.}\DecValTok{5}\NormalTok{),}
        \DataTypeTok{axis.text.y =} \KeywordTok{element_text}\NormalTok{(}\DataTypeTok{colour=}\StringTok{"grey40"}\NormalTok{, }\DataTypeTok{size=}\DecValTok{12}\NormalTok{),}
        \DataTypeTok{strip.text =} \KeywordTok{element_text}\NormalTok{(}\DataTypeTok{size=}\DecValTok{14}\NormalTok{),}
        \DataTypeTok{text =} \KeywordTok{element_text}\NormalTok{(}\DataTypeTok{size=}\DecValTok{16}\NormalTok{))}
\end{Highlighting}
\end{Shaded}

\includegraphics{R-data-wrangling_files/figure-latex/tilted-xlabels-1.pdf}

If you like the changes you created better than the default theme, you can save them as an object to be able to easily apply them to other plots you may create:

\begin{Shaded}
\begin{Highlighting}[]
\NormalTok{grey_theme <-}\StringTok{ }\KeywordTok{theme}\NormalTok{(}\DataTypeTok{axis.text.x =} \KeywordTok{element_text}\NormalTok{(}\DataTypeTok{colour=}\StringTok{"grey40"}\NormalTok{, }
                                               \DataTypeTok{size=}\DecValTok{12}\NormalTok{, }\DataTypeTok{angle=}\DecValTok{90}\NormalTok{, }
                                               \DataTypeTok{hjust=}\NormalTok{.}\DecValTok{5}\NormalTok{, }\DataTypeTok{vjust=}\NormalTok{.}\DecValTok{5}\NormalTok{),}
                   \DataTypeTok{axis.text.y =} \KeywordTok{element_text}\NormalTok{(}\DataTypeTok{colour=}\StringTok{"grey40"}\NormalTok{, }\DataTypeTok{size=}\DecValTok{12}\NormalTok{),}
                   \DataTypeTok{text=}\KeywordTok{element_text}\NormalTok{(}\DataTypeTok{size=}\DecValTok{16}\NormalTok{))}

\KeywordTok{ggplot}\NormalTok{(}\DataTypeTok{data =}\NormalTok{ Yazoo_stops, }\KeywordTok{aes}\NormalTok{(}\DataTypeTok{x =}\NormalTok{ violation, }\DataTypeTok{y =}\NormalTok{ driver_age)) }\OperatorTok{+}
\StringTok{  }\KeywordTok{geom_boxplot}\NormalTok{() }\OperatorTok{+}\StringTok{ }
\StringTok{  }\NormalTok{grey_theme}
\end{Highlighting}
\end{Shaded}

\includegraphics{R-data-wrangling_files/figure-latex/save-reapply-theme-1.pdf}

Note that it is also possible to change the fonts of your plots. If you are on Windows, you may have to install the \href{https://github.com/wch/extrafont}{\textbf{extrafont} package}, and follow the instructions included in the README for this package.

\begin{quote}
Challenge

With all of this information in hand, please take another five minutes to either
improve one of the plots generated in this exercise or create a beautiful graph
of your own. Use the RStudio \href{https://www.rstudio.com/wp-content/uploads/2016/11/ggplot2-cheatsheet-2.1.pdf}{\textbf{\texttt{ggplot2}} cheat sheet} for
inspiration.
\end{quote}

\begin{quote}
Here are some ideas:
\end{quote}

\begin{quote}
\begin{itemize}
\tightlist
\item
  See if you can change the thickness of the lines.
\item
  Can you find a way to change the name of the legend? What about its labels?
\item
  Try using a different color palette (see \url{http://www.cookbook-r.com/Graphs/Colors_(ggplot2)/}).
\end{itemize}
\end{quote}

After creating your plot, you can save it out to a file in your prefered format. You can change the dimension (and resolution) of your plot by adjusting the appropriate arguments (\texttt{width}, \texttt{height} and \texttt{dpi}):

\begin{Shaded}
\begin{Highlighting}[]
\NormalTok{my_plot <-}\StringTok{ }\NormalTok{stops_facet_plot }\OperatorTok{+}
\StringTok{  }\KeywordTok{labs}\NormalTok{(}\DataTypeTok{title =} \StringTok{'Observed violations per day of week'}\NormalTok{,}
         \DataTypeTok{x =} \StringTok{'Weekday of observation'}\NormalTok{,}
         \DataTypeTok{y =} \StringTok{'Number of violations'}\NormalTok{) }\OperatorTok{+}
\StringTok{  }\KeywordTok{theme_bw}\NormalTok{() }\OperatorTok{+}\StringTok{ }
\StringTok{  }\KeywordTok{theme}\NormalTok{(}\DataTypeTok{axis.text.x =} \KeywordTok{element_text}\NormalTok{(}\DataTypeTok{colour=}\StringTok{"grey40"}\NormalTok{, }\DataTypeTok{size=}\DecValTok{12}\NormalTok{, }\DataTypeTok{angle=}\DecValTok{90}\NormalTok{, }\DataTypeTok{hjust=}\NormalTok{.}\DecValTok{5}\NormalTok{, }\DataTypeTok{vjust=}\NormalTok{.}\DecValTok{5}\NormalTok{),}
        \DataTypeTok{axis.text.y =} \KeywordTok{element_text}\NormalTok{(}\DataTypeTok{colour=}\StringTok{"grey40"}\NormalTok{, }\DataTypeTok{size=}\DecValTok{12}\NormalTok{),}
        \DataTypeTok{strip.text =} \KeywordTok{element_text}\NormalTok{(}\DataTypeTok{size=}\DecValTok{14}\NormalTok{),}
        \DataTypeTok{text =} \KeywordTok{element_text}\NormalTok{(}\DataTypeTok{size=}\DecValTok{16}\NormalTok{))}

\KeywordTok{ggsave}\NormalTok{(}\StringTok{"MS_weekday_stops_facets.png"}\NormalTok{, my_plot, }\DataTypeTok{width=}\DecValTok{15}\NormalTok{, }\DataTypeTok{height=}\DecValTok{10}\NormalTok{)}
\end{Highlighting}
\end{Shaded}

Note: The parameters \texttt{width} and \texttt{height} also determine the font size in the saved plot.

\bibliography{book.bib,packages.bib}


\end{document}
